\documentclass[../main.tex]{subfiles}

\begin{document}

\section{Multiplication operators and infinite matrices}
In this section, we will focus on multiplication operators on $L^2$ and
discrete operators on $\ell^2$.\footnote{We will take the convention that
$L^2$ denotes $L^2(\Omega)$ for an arbitrary implicit measure space $\Omega$,
and similarly for $\ell^2$.}
Operators on $\ell^2$ have intuitive
interpretations for approximation, where we can treat them as 'infinite
matrices' that we truncate by 'chopping off' the matrix beyond a certain number
of rows or columns. While multiplication operators are not discrete in
themselves, we will also see that their pollution can be concretely explained
via a certain type of infinite matrix operator.

\subsection{The spectrum of a multiplication operator}
\begin{definition}[\textbf{Multiplication operator}]
\index{multiplication operator}
Let $\hilbert$ be a Hilbert space. For a given $a \in \hilbert$, the
multiplication operator $M_a$ on $\hilbert$ is defined by the action
$M_af(x) = a(x)f(x)$; that is, it acts by pointwise multiplication with
$a$. We call $a$ the `symbol' of the multiplication operator.
\end{definition}

We can see immediately that the adjoint of $M_a$ is $M_{\overline{a}}$, and thus
that $M_a$ is self-adjoint iff $a$ is real-valued.

The spectrum of a multiplcation operator is easy to calculate. We first
need to define the 'essential range' of a function. Intuitively, this is
similar to the standard range of a function, but ignoring values taken by
the function on a set of measure zero - two functions which are equal
almost everywhere will have the same essential range.

\begin{definition}[\textbf{Essential range and essential supremum}]
\label{defn:essential-range}\index{essential range}
The essential range of a real-valued function $f$ is the set:
 $$\{k \in \mathbb{R} : \forall \varepsilon > 0, \mu\{x : |f(x) - k| < \varepsilon\} > 0\}$$
where $\mu$ is the Lebesgue measure.
  
The essential supremum of $f$, denoted $\mathrm{esssup}f$, is the supremum of
the essential range of $f$. We define $\mathrm{essinf}f$ \emph{mutatis
mutandis} for the infimum.
\end{definition}

The normed vector space $L^\infty$ is the space of all bounded measurable
functions, and its norm is
 $$\|f\|_\infty = \text{esssup}|f|.$$

\begin{lemma}
\label{thm:mult-symbol-char}
(Adapted from \cite{arveson2002short})
  Let $\Omega$ be a $\sigma$-finite measure space.
  A multiplication operator on $L^2(\Omega)$ is bounded if and only if its symbol is in
  $L^\infty(\Omega)$.
\end{lemma}
\begin{proof}
Let $M_a$ be a multiplication operator on $L^2$ with symbol
$a \in L^\infty$. We see that $|f(z)| \leq \|a\|_\infty$ for almost all
$z$, so for any $g \in \hilbert$, $|a g| = |a||g| \leq \|a\|_\infty |g|$
pointwise almost everywhere; thus
\begin{equation*}
\|a g\|_2 = \sqrt{\int |a g|^2} \leq \sqrt{\int \|a\|_\infty^2|g|^2} 
	  = \|a\|_\infty \sqrt{\int |g|^2} = \|a\|_\infty \|g\|_2.
\tag{$\star$}
\end{equation*}
This implies $\|M_a\| = \sup_{g \in \hilbert} \|a g\|_2 \leq \|a\|_\infty$, so $M_a$ is bounded.

Conversely, assume $M_a$ is bounded, $a \neq 0$ and let $c < \|a\|_\infty$ (of
course, we do not assume $\|a\|_\infty$ is finite). Then we know that
the set $\{z : |a(x) > c|\}$ must have positive measure, and by
$\sigma$-finiteness we have a subset $E \subseteq \{z : |a(x) > c|\}$,
and its characteristic 
function $\1_E$ is in $L^2$. Then $|a \1_E| \geq c \1_E$, and by a similar
calculation to ($\star$) and taking the supremum, $\|M_a  \1_E\|_2 \geq c
\|\1_E\|_2$, so $\|M_a\| \geq c$. Taking the supremum over all $c$ gives
$\|M_a\| \geq \|a\|_\infty$, so $\|a\|_\infty$ is finite and so $a \in
L^\infty$. \end{proof}

Note from this lemma we also get that $\|M_a\| = \|a\|_\infty$.

\begin{theorem}[\textbf{Spectrum of a multiplication operator}]
  \label{thm:mult-op-spec}
  The spectrum of the multiplication operator $M_a$ is the essential range of its symbol $a$.
\end{theorem}
\begin{proof}
(Adapted from \cite{garcia2023operator})
Let $(M_a - \lambda)f = g$. Then $f(z) = \frac{g}{a(z) - \lambda}$, and we see
that $(M_a - \lambda)$ is invertible if and only if the operator 
$M_{(a - \lambda)^{-1}}$ is bounded, which is the case if and only if
$(a - \lambda)^{-1} \in L^\infty$ by Lemma \ref{thm:mult-symbol-char}.
This is the case exactly when $(a - \lambda) \geq \epsilon$ almost everywhere -
by definition, this means that $(M_a - \lambda)$ is invertible if and
only if $\lambda$ is not in the essential range of $a$.
\end{proof}

We will now estimate this known spectrum with a Ritz matrix to see how
it is affected by spectral pollution.

\begin{example}
  Let $M_f$ be the multiplication operator on $L^2[0, 1]$ with symbol
$$
f: x \mapsto \begin{cases}
x & x < 1/2 \\
x + 1/2 & \text{otherwise.}
\end{cases}
$$
By Theorem \ref{thm:mult-op-spec}, the spectrum of $M_f$ is the set $[0, 1/2]
\cup [1, 3/2]$. If our Ritz approximation works as expected, we should
see the approximation create two dense clusters of eigenvalues which
approach these intervals.
\end{example}

Figure \ref{fig:mult-op-spec} shows a plot of the approximate spectrum for
various Ritz matrix sizes. This approximation was done for the sequence of
truncations $T_{25n}$ for $n \in \{2, 3, \hdots, 20\}$ over the orthonormal
basis $\phi_n(x) = \exp(2 i \pi n x)$ for $n \in \mathbb{Z}$; each truncation
was over the space $\mathrm{span}\{\exp(2 i \pi k x), k \in \mathbb{Z}, |k| < n/2\}$.

We do successfully and rather quickly get eigenvalues corresponding to our
actual spectrum, but we also get a lot of other eigenvalues; some of them are
converging to points in the spectrum, but as the approximation improves, the
pollution doesn't fully dissipate.

\begin{figure}[h]
\centering
\includegraphics[scale=0.5]{mult-op-spec.png}
\caption{The approximate spectrum of the multiplication operator $M_f$ for Ritz
	matrices of increasing size, with an orthonormal basis of exponential
  functions (as explained above).
  The shaded green areas correspond to the
	actual spectrum of the operator.}
\label{fig:mult-op-spec}	
\end{figure}

Some of these extra values seem to eventually converge into the correct part of
the spectrum, but other values in the central gap seem to stay around.
Of course, this is not particularly rigorous - we do not yet know anything about
the asymptotic behaviour of the eigenvalues of these approximations (it may well
converge far beyond our biggest estimate here of a $500 \times 500$ matrix) -
but this example is certainly motivating. If one had no knowledge of the actual
spectrum of $M_f$, they would be forgiven for considering this to be strong
empirical evidence that the spectrum contains the range $[0.8, 1.0]$ or at least
some subset of it.

Indeed, we can see the effect of using different orthonormal bases here: if we
use the orthonormal basis $\phi_n(x) = {\sqrt{2}}\sin(\pi nx)$, the
pollution appears in different places (plotted in Figure \ref{fig:mult-op-sin}).

\begin{figure}[h]
\centering
\includegraphics[scale=0.5]{mult-op-sin.png}
\caption{The approximate spectrum of the multiplication operator $M_f$ for Ritz
	matrices of increasing size, using the orthonormal basis of sines,
  as explained above.
  The shaded green areas correspond to the
	actual spectrum of the operator. Note that the pollution appears in different
  places to where it appears in Figure \ref{fig:mult-op-spec}.}
\label{fig:mult-op-sin}	
\end{figure}

These numerical examples raise a question about the nature of this
pollution - why are we only getting it in the gap between the intervals, and not
far outside?

For multiplication operators, there is an
an underlying structure to its Ritz matrices which
make it possible to concretely identify the existence and location of spectral
pollution. Following chapters will then devise technology that allows us to 
answer these questions in greater generality.

If we take a closer look at the structure of our approximating Ritz matrices we
uncover that they have a 'banded' structure.
\begin{example}\label{exp:mult-op-toeplitz}
Let $M_f$ be a multiplication operator on $L^2[0, 1]$. Now we create the Ritz
matrix, $A_{j,k} = (M_f \phi_j, \phi_k)$, choosing the orthonormal basis
$\phi_n(x) = \exp(2 \pi i n x).$

We now note that there is a structure to our matrix:
\begin{align*}
(M_f \phi_j, \phi_k) & = \int_0^1 f(x) \exp(2 \pi i j x) \exp(-2 \pi i k x) dx \\
& = \int_0^1 f(x) \exp(2 \pi i (j-k) x)\\
& = c_{j-k}
\end{align*}
where $c_n$ is the n'th Fourier coefficient of $f$. Thus our Ritz matrix depends only
on the value of $j-k$; in particular, it is constant along each
diagonal. This is a special type of matrix known as a Toeplitz matrix.
\end{example}
As a result, the approximation of our operator $M_f$ is equal to the
approximation of a infinite matrix $(T_f)_{j,k} = c_{j-k}, j,k \in \mathbb{N}_0$
by its truncations $(T_{f,n})_{j,k} = c_{j-k}$ for $j, k \leq n$. And nowhere in
this derivation did we use particular
properties of $f$; we can repeat this reasoning with any function capable of
being represented by a Fourier series. Let us systematise what we have seen.

\subsection{Toeplitz operators}\label{sec:toeplitz}

A full account of Toeplitz theory is the subject
of whole monographs (such as \parencite{bottcher1990analysis}) and is an entire
subfield in itself; here we will stick to exploring their spectra, and how these
relate to the spectra of their multiplication operator neighbours. The following
definitions are adapted from \parencite{arveson2002short}.

\begin{definition}[\textbf{Toeplitz matrix}]\index{Toeplitz matrix}
  A matrix $A$ (finite or infinite) is Toeplitz if it is constant along its
  diagonals; that is, $A_{i,j} = A_{i+1,j+1}$ for any $i, j$ (where $i,j <
  N \in \mathbb{N}$ if $A$ is finite, or $i,j \in \mathbb{Z}_+$ when $A$
  is infinite).
\end{definition}

\begin{definition}[\textbf{Hardy space}]\index{Hardy space}
  Let $\zeta$ be the monomial function on $L^2(\mathbb{T})$, $\zeta(z) = z$, where
  $\mathbb{T}$ is the unit circle. Then the Hardy space $H^2$ is the span
  of all non-negative exponents of $\zeta$; $\textrm{span}\{1, \zeta,
  \zeta^2 \hdots\}$.
\end{definition}

As we are on the unit circle, $z$ is more familiarly $e^{i \theta}$ for some
$\theta$; then the basis of Hardy space becomes $\{e^{i n \theta}\}_{n \in
\mathbb{Z}_+}$. Then we can identify any element $f \in H^2$ as any
square-integrable function defined on the unit circle with the Fourier series
$$f(e^{i \theta}) \sim \sum_{k=0}^\infty c_k e^{i k \theta},$$

that is, with all negative Fourier coefficients equal to zero. This can be
identified with the infinite matrix on $\ell^2(\mathbb{Z}+)$ via the isomorphism
$\sum_{k=0}^\infty c_k e^{i k \theta} \mapsto (c_k)_{k \in \mathbb{Z}_+}$
\parencite{bottcher1990analysis}.

A generalised definition can be made for $H^p$ via any $L^p(\mathbb{T})$.
However, in the literature, $H^2$ is often referred to as \emph{the}
Hardy space.

\begin{definition}[\textbf{Toeplitz operator}]\index{Toeplitz operator}
  Let $\phi \in L^{\infty}(\mathbb{T})$ be bounded and measurable on the unit
  circle. The Toeplitz operator $T_\phi$ is the compression of $M_\phi$ to
  the Hardy space: $T_{\phi} = P_{H^2} M_\phi \big|_{H^2}$. We call $\phi$
  the `symbol' of $T_\phi$.
\end{definition}

One may notice what appears like a clash of notation between the induced
Toeplitz matrix and the Toeplitz operator. In fact, there is a correspondence
between the matrix on $\ell^2(\mathbb{Z}_+)$ and the operator on $L^2(\mathbb{T})$.

\begin{theorem}
  Let $T_\phi$ be a Toeplitz operator. Then the matrix representation of $T_\phi$
  is a Toeplitz matrix where the $n,m$'th entry is the $m-n$'th Fourier coefficient
  of $\phi$.
\end{theorem}
\begin{proof}\cite{garcia2023operator}
First, note that $\{\zeta^n\}_{n\in\mathbb{N}}$ is an orthonormal basis of
the Hardy space (by definition).
Then the $n,m$th entry of the matrix representation of the operator $T_\phi$ is given
  \begin{align*}
    (T_{\phi}\zeta^n, \zeta^m) & = (P_{H^2}(\phi\zeta^n), \zeta^m) & \\
                               & = (\phi\zeta^n, P_{H^2}\zeta^m) 
                                 & \text{\emph{(as $P_{H^2}$ is self-adjoint)}} \\
                               & = (\phi\zeta^n, \zeta^m) &\\
                               & = \int_{\mathbb{T}} \phi\zeta^n \zeta^{-m}
                                 & \text{\emph{(by defn. of inner product on $L^2$)}}\\
                               & = \int_{\mathbb{T}} \phi \zeta^{n-m} &\\
  \end{align*}
which is the $m-n$'th Fourier coefficient of $\phi$.
\end{proof}
The opposite is also true; every Toeplitz matrix admits a symbol $\phi$ which
we can use to create a Toeplitz operator \cite{garcia2023operator}.
Many properties of Toeplitz operators are hard to see via infinite matrices.
Being able to represent them as both Fourier series and as the  compressions of
multiplication operators puts us on the firmer ground of functional analysis,
rather than asymptotic linear algebra. From this, we
are now in the position to exactly calculate the spectrum of a Toeplitz operator
with real-valued symbol.

To begin, we require a pair of properties regarding functions in
$L^1(\mathbb{T})$'s Hardy space, $H^1$. 

\begin{lemma}[\textbf{Properties of Hardy functions}]
\label{thm:h1-properties}
  Let $H^1$ be the space of all functions $f \in L^1(\mathbb{T})$ where $f$ has
  the Fourier series
  $$f(e^{i \theta}) \sim \sum_{n=0}^\infty a_n e^{i n \theta}$$
  i.e. has no negative Fourier coefficients. Then the following properties hold:
  \begin{enumerate}
    \item\label{item:h2-products-in-h1} If $f, g \in H^2$, then $fg \in H^1$;
    \item\label{item:f-h1-const} If $f \in H^1$ is real-valued, then $f$ is constant.
  \end{enumerate}
\end{lemma}
\begin{proof}
  Firstly, we note a property of Fourier series which will be used in both proofs:
  Let the function $f$ have Fourier coefficients $c_n$, and its complex conjugate
  $\overline{f}$ have the Fourier coefficients $d_n$. Then $d_n = \overline{c_{-n}}:$
  \begin{equation}
  \label{eqn:fourier-conjugates}
    \overline{c_{-n}} = \overline{\frac{1}{2\pi} \int_{-\pi}^{\pi}  f(x) e^{inx} dx} 
		    = \frac{1}{2\pi} \int_{-\pi}^{\pi} \overline{f(x)} e^{-inx} dx
        = d_n.
  \end{equation}
  In particular, if $f$ is real-valued, we have $c_n = \overline{c_{-n}}$.

  (\ref{item:h2-products-in-h1}.) Let $f, g$ be in $H^2$. Then in particular, $f,
  g \in L^2$, and so their product $fg$ is in $L^1$ (this can be seen
  directly by the H\"older inequality).

  Now if $f$ has Fourier coefficients $a_k$ and $g$ has Fourier coefficients $b_k$,
  the $k'th$ entry of the product of their Fourier series is
  \begin{equation}
  \label{eqn:prod-fourier-series}
    \sum_{n \in \mathbb{Z}} a_n b_{k-n}
  \end{equation}
  For negative $k$, 
  we now have that either $n < 0$ so $a_n = 0$, or $n \geq 0$ so $b_{k-n} = 0$
  as $k-n < 0$; this means that the product of the Fourier series satisfies
  the Hardy space property. However, must justify that this reflects the behaviour
  of the original functions, i.e. that in this case, the Fourier series of a product
  is equal to the product of the Fourier series. This is true in $H^2$, via
  Parseval's theorem \cite{halmos1982hilbert}:
  
  Parseval's theorem states that for any two square-integrable, $2\pi$-periodic functions $U, V$,
  $$\frac{1}{2\pi} \int_{-\pi}^{\pi} U(x)\overline{V(x)} dx = \sum_{n \in \mathbb{Z}}u_n \overline{v_n}$$
  where $u_n$, $v_n$ are the Fourier coefficients of $U$ and $V$ respectively.

  All functions in $H^2$ are $2\pi$-periodic and square-integrable by definition.
  The $n$'th Fourier coefficient of $fg$ is given by
  $$\frac{1}{2\pi} \int_{-\pi}^{\pi} f(x) g(x) e^{-inx}.$$
  Then we apply this to Parseval's theorem, with $U = f$ and $V = \overline{g e^{-inx}}$.
  Note that the $k$'th Fourier coefficient of $\overline{g e^{-inx}}$ is given by
  $$\frac{1}{2\pi} \int_{-\pi}^{\pi} \overline{g(x)} e^{-i(k-n)x} = b'_{n-k}$$
  where $b'_n$ is the $n$'th Fourier coefficient of $\overline{g}$.

  Then applying Parseval's theorem,
  $$\frac{1}{2\pi} \int_{-\pi}^{\pi} f(x) g(x) e^{-inx} = \sum_{n \in \mathbb{Z}}a_n \overline{b'_{n-k}}$$
  which equals the series of equation (\ref{eqn:prod-fourier-series})
  as $\overline{b'_{n-k}} = b_{k-n}$ by (\ref{eqn:fourier-conjugates}).

(\ref{item:f-h1-const}.) As $f$ is in Hardy space, $c_{-n} = 0$ for all $n \in \mathbb{N}$, and so $c_n =
  \overline{c_{-n}} = 0$ for all $n \in \mathbb{N}$, by equation (\ref{eqn:fourier-conjugates}).
  Thus the only coefficient remaining is $c_0$, and a function with a constant Fourier series
  is a constant function. (This proof is valid for any $H^p, p \in [1, \infty)$.)
\end{proof}

\begin{lemma}
\label{thm:toeplitz-self-adjoint}
  Let $\phi \in L^\infty$ be real-valued; then the Toeplitz operator $T_\phi$ is self-adjoint. 
\end{lemma}
\begin{proof}
Let $u, v \in H^2$. Then 
\begin{align*}
(T_\phi u, v) & = ( P_{H^2}M_\phi u, v) & \\
& = (M_\phi u, v) & \text{\emph{as $P_{H^2}$ is self-adjoint and $P_{H^2}v = v$}}\\
& = (u, M_\phi v) = (P_{H^2}u, M_\phi v) = (u, T_\phi v),
\end{align*}
using that $M_\phi$ is self-adjoint if $\phi$ is real-valued.
\end{proof}

We are now in the position to calculate the form of the spectrum for any
Toeplitz operator. This spectrum was first found by Toeplitz himself under a
stronger set of regularity assumptions, and then weakened by Wiener via his
Tauberian theorem \parencite{schmidt1960toeplitz}.
Our proof will consist of a series of claims about the resolvent set of
$T_\phi$, following a proof outlined in an exercise of Arveson
(\parencite{arveson2002short}, Chapter 4.6, exercises 2-5) which is itself based
on a proof by Hartman and Wintner.

\begin{theorem}[\textbf{Hartman-Wintner}]\index{Hartman-Wintner}\label{thm:hartman-wintner}
Let $\phi \in L^\infty$ be real-valued. Then $\Spec(T_\phi) = [m, M]$, where $m$
and $M$ are the infimum and supremum of the essential range of $\phi$ respectively.
\end{theorem}
\begin{proof}
Note we already know by Lemma \ref{thm:toeplitz-self-adjoint} that as $\phi$ is
real-valued, $T_\phi$ is self-adjoint, and so its spectrum is on the
real line. Furthermore, we will assume that $\phi$ is non-constant, as
if $\phi$ is constant then $T_\phi$ is some constant multiple of the
identity operator and its spectrum is simply the set containing that constant,
and the result holds. 
\begin{enumerate}[I.]
\item Let $\lambda \in \mathbb{R}$ be such that $T_\phi - \lambda$ is
invertible. Then for some $k \in \mathbb{C}$ there is a non-zero
function $f \in H^2$ such that $(T_\phi - \lambda)f(z) = k$ for
any $z$. By the definition of invertibility this is true for $f$
= $(T_\phi - \lambda)^{-1}\kappa$, where $\kappa$ is the
constant function in $H^2$; $\kappa : z \mapsto k$.

\item Now we claim that $(\phi - \lambda)|f|^2$ is constant almost everywhere.
To do this, we first show that it is in $H^1$. Indeed, we have
$(\phi - \lambda)|f|^2 = ((\phi - \lambda)\overline{f}) f$.
Then by the previous part, 
$$(\phi - \lambda)\overline{f} = \overline{(\phi - \lambda)f} = \overline{\kappa}$$
where $\overline{\kappa}$ maps $z$ to $\overline{k}$. Then it is still a
constant function so is still in $H^2$, and by Lemma
\ref{thm:h1-properties}.\ref{item:h2-products-in-h1},
$((\phi - \lambda)\overline{f}) f$ is in $H^1$ as a product of two $H^2$ functions.
Now we use Lemma \ref{thm:h1-properties}.\ref{item:f-h1-const}: because $\phi$ and
$\lambda$ are real-valued, $(\phi - \lambda)|f|^2$ is also real-valued, so must
be constant. Let this constant be called $c$.
\item Our next claim is that $\phi - \lambda$ crosses the $x$-axis almost nowhere.
This is almost immediate once we invoke a theorem of F. and M. Riesz\footnote{Which can
be proven as a corollary of a famous theorem of Buerling; see
\parencite{arveson2002short}, chapter 4.5.}, which states
\begin{displayquote}
\emph{Let $f$ be a non-zero function in $H^2$. Then the set $\{z \in \mathbb{T}
: f(z) = 0\}$ has Lebesgue measure zero.}
\end{displayquote}
This means that $(\phi(z) - \lambda) = \frac{c}{|f(z)|^2}$ is well-defined on
$L^\infty$. Then because $|f|^2 > 0$ a.e., $(\phi - \lambda)$ is
positive almost everywhere if $c > 0$, and negative almost
everywhere if $c < 0$. Note that if $c = 0$, then $(\phi(z) -
\lambda) = 0$, so $\phi$ is the constant function with value
$\lambda$ and the result holds by our remark at the beginning of
this proof.
\item Finally, we show $\Spec(T_\phi) = [m, M]$. By our previous claim, if
$T_\phi - \lambda$ is invertible, then either:
\begin{itemize}
\item $\phi(z) - \lambda < 0$ a.e., so $\phi(z) < \lambda$ a.e., so $\lambda > M$, or
\item $\phi(z) - \lambda > 0$ a.e., so by the same argument $\lambda < m$.
\end{itemize}
Thus $T_\phi - \lambda$ is invertible only outside of the set $[m, M]$, as required.
\end{enumerate}
\end{proof}

\subsection{Multiplication operators, Toeplitz operators, and spectral pollution}
Let us now bring the discussion back to spectral pollution, and to our discovery
in Example \ref{exp:mult-op-toeplitz}. The Ritz matrices corresponding to the
multiplication operator $M_f$ is \emph{identical} to the Ritz matrices
corresponding to the Toeplitz operator $T_f$ (which are just truncations of the 
corresponding Toeplitz matrix), but these operators have different spectra. No
wonder we are seeing extra eigenvalues in the gap; the approximation `cannot
tell' the difference between an operator which has the essential range as its
spectrum from one which has the same maximum and minimum but with
all gaps filled in. Heuristically, one may even expect that with a large enough
approximation, the \emph{entire gap} could fill up with spurious
eigenvalues\footnote{We also have the inverse question; if we are approximating
the Toeplitz operator $T_f$, why does it take so much longer to 
approximate the spectrum in the interval $\Spec(T_f) \setminus \Spec(M_f)$?}.\\

More rigorously, it is possible to show that for any point in a gap of the
multiplication operator's spectrum, some subsequence of truncations will have an
approximate eigenvalue converging to that point. 

\begin{theorem}[\textbf{Schmidt-Spitzer \parencite{schmidt1960toeplitz}}]
\index{Schmidt-Spitzer}
Let $T_n$ be the $n \times n$ matrix created by taking the first $n$ rows and
columns of a Toeplitz matrix $T_\phi$. Consider the set $$B =
\{\lambda \in \mathbb{C} : \lambda = \lim_{m \rightarrow \infty}
\lambda_{i_m}, \lambda_{i_m} \in \Spec(T_{i_m}), i_m \rightarrow \infty
\}.$$
where $i_m$ is a subsequence of $\mathbb{N}$. Then if $\phi$ is real-valued, $B = \Spec(T_\phi)$.
\end{theorem}
\begin{proof}
If $\phi$ is real-valued, then by Lemma \ref{thm:toeplitz-self-adjoint} we can
restrict our attention to the real line. Label the eigenvalues of $T_n$
as $\lambda_{n_1}, \hdots \lambda_{n_{n+1}}$. 
 
We now invoke a theorem of Szeg\H{o} \cite{grenander2001toeplitz} regarding the
distribution of eigenvalues: on any real interval $[a, b]$,
\begin{equation}
\label{eqn:szego-eigvals}
\lim_{n \rightarrow \infty} \frac{1}{n+1} \sum_{i=1}^{n+1} \1_{[a, b]}(\lambda_{n_i}) 
= \frac{1}{2 \pi} \mu\{\theta : a \leq \phi(e^{i \theta}) \leq b\}
\end{equation}
where $\mu$ is the Lebesgue measure, and $\1$ the characteristic function of $[a,b]$ 
(i.e. $\1_{[a, b]}(\lambda_{n_i})$ is 1 if $\lambda_{n_i} \in [a, b]$ and zero
otherwise). Note that this makes sense, as $T_n$ is Hermitian and
therefore every eigenvalue is real-valued. Note also that by Theorem
\ref{thm:hartman-wintner} and the definition of the unit circle
$\mathbb{T}$,
the spectrum of $T_\phi$ is $\{\phi(e^{i \theta}): \theta \in [0, 2\pi]\}$. The
claim of equation (\ref{eqn:szego-eigvals}) is therefore ``as the size
of the Toeplitz matrix increases, the proportion of eigenvalues of $T_n$
in $[a, b]$ converges to the proportion of the spectrum in $[a, b]$''. 
Of course, this proportion will never be 0 for any interval $[a, b] \subseteq
[\textrm{essinf}\phi, \textrm{esssup}\phi]$ of positive measure, so we
can guarantee that a sequence of truncations has an eigenvalue which
converges in $[a, b]$. 

Now consider a point $\xi \in \Spec(T_\phi)$. For each $m \in \mathbb{N}$, we
know by the above that we can choose $i_m \in \mathbb{N}$ such that
there is an eigenvalue $\lambda_{i_m}$ in the interval $[\xi -
\frac{1}{m}, \xi + \frac{1}{m}]$, and therefore $|\xi - \lambda_{i_m}| <
\frac{1}{m}$. Then as $m \rightarrow \infty$, $i_m \rightarrow \infty$,
and thus $\xi \in B$, giving $\Spec(T_\phi) \subseteq B$.

Conversely, suppose we have some value $\xi \notin \Spec(T_\phi)$. If $\xi <
\mathrm{essinf}\phi$, then $\xi$ is contained in some interval $[a,
\mathrm{essinf}(\phi) - \varepsilon]$ for $\varepsilon > 0$ and $a < \xi$. But
then by equation \ref{eqn:szego-eigvals}, the number of eigenvalues
$\lambda_{n_i}$ in $[a, \mathrm{essinf}(\phi) - \varepsilon]$ must converge to
zero, so there is no sequence of eigenvalues of truncations which converge to
$\xi$. Thus $\xi \notin B$. The same argument can be made \emph{mutatis
mutandis} for $\xi > \mathrm{esssup}\phi$, giving $B \subseteq \Spec(T_\phi)$ as
required.
\end{proof}

This is a striking result - for any point in the spectral gap $\Spec(T_\phi)
\setminus \Spec(M_\phi)$ (or indeed in $\Spec(M_\phi)$, but they aren't really
polluting anything), we can choose a sequence of truncations such that there is
guaranteed to be spectral pollution at that point. It also shows us that the
Toeplitz operator itself does not have any pollution for real symbol.
In Chapter \ref{chapter:bounds}, we will see that this is generally true for
a wider variety of operators, albeit not so concretely.
\end{document}
