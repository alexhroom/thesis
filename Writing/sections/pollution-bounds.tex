\documentclass[../main.tex]{subfiles}

\begin{document}

\section{Bounding spectral pollution}\label{chapter:bounds}

\subsection{The essential spectrum of an operator}\label{sec:ess-spec}

We begin by producing a result which allows us to constructively classify points
in the spectrum of an operator.
\begin{theorem}[\textbf{Approximate eigenvalue theorem}]
\label{thm:approx-eigenvalue-thm}
  Consider an operator $T$ on a Hilbert space $\hilbert$, and $\lambda \in
  \mathbb{C}$. $\lambda$ is in the spectrum of $T$ if there exists a
  sequence $u_n$ in $\hilbert$ with the following properties:
  \begin{itemize}
  \item $\|u_n\| = 1\quad \forall n \in \mathbb{N}$, and
  \item $\lim_{n \rightarrow \infty}\|(T - \lambda)u_n\|  \rightarrow 0$.
  \end{itemize}
\end{theorem}
\begin{proof}
Assume for contradiction that the resolvent $(T - \lambda)^{-1}$ exists. Then:
\begin{align*}
0 \leq \lim_{n \rightarrow \infty} \|u_n\| & 
	= \lim_{n \rightarrow \infty} \|(T - \lambda)^{-1}(T - \lambda)u_n\| & \\
& \leq \|(T - \lambda)^{-1}\| \lim_{n \rightarrow \infty} \|(T - \lambda)u_n\| & 
	\text{\emph{(as $(T - \lambda)^{-1}$ is bounded)}}\\
& = 0
\end{align*}
and so $\|u_n\| \rightarrow 0$. But $\|u_n\| = 1$ for every $n$, so it cannot
converge to zero! Thus this bounded inverse does not exist.
\end{proof}
We will call the subset of the spectrum created by this theorem the
\textbf{approximate point spectrum}\index{approximate point spectrum}, denoted
$\Spec_{ap}$:
$$\Spec_{ap}(T) \eqdef \{ \lambda \in \mathbb{C} : \exists u_n \text{ s.t. } \|u_n\| = 1\quad \forall n \in \mathbb{N},\text{ and }\lim_{n \rightarrow \infty}\|(T - \lambda)u_n\|  \rightarrow 0\}.$$
Note that any eigenvalue is in the approximate point spectrum - if $\varphi$ is
a normalised eigenvector corresponding to $\lambda$, then the constant sequence
$u_n = \varphi$ satisfies this property. Indeed, for certain types of operator,
the entire spectrum consists of approximate eigenvalues.

\begin{lemma}
  For any operator $T$, $\Spec(T) = \overline{\Spec_{ap} (T^*)} \cup \Spec_{ap} (T)$.
\end{lemma}
\begin{proof}
Consider what makes a point $\lambda$ capable of being in $\Spec(T) \setminus
\Spec_{ap}(T)$. $(T - \lambda)$ must be injective (else $(T - \lambda)u
= (T - \lambda)v$ for some $u \neq v$ and so $u-v$ is an eigenfunction)
and $\Ran(T - \lambda)$ must not be dense in $\hilbert$ (else it is
surjective and thus bijective, or can be extended to an operator which
is surjective and thus bijective).

We can prove that for any operator $T$, $\Ran(T)^\perp = \Ker(T^*)$:
\begin{align*}
\text{ for } u \in \Ran(T), v \in \Ker(T^*), (u, v) & = (Tw, v) \text{ for some $w \in \hilbert$} \\
& = (w, T^*v) = (w, 0) = 0 \\
& \Rightarrow \Ker(T^*) \subseteq \Ran(T)^\perp ;\\
x \in \Ran(T)^\perp & \Rightarrow (Aw, x) = 0, \\
& \Rightarrow (w, A^*x) = 0\ \forall w \in \hilbert \\
&  \Rightarrow A^*x = 0 \\
& \Rightarrow \Ran(T)^\perp \subseteq \Ker(T^*).
\end{align*}

Then as there is some element $\eta$ not in the closure of  $\Ran(T - \lambda)$,
by the Projection Theorem $\eta \in \Ker(T^* - \overline{\lambda})$; hence $(T^*
- \overline{\lambda})\eta = 0$ and so $\eta$ is an eigenvector for $T^*$ with
eigenvalue $\overline{\lambda}.$ \end{proof}

\begin{theorem}\label{thm:normal-spec}
\cite{halmos1982hilbert}
  Let $T$ be a normal operator (i.e. it commutes with its adjoint; $TT^* = T^*T$).
  Then $$\Spec(T) = \Spec_{ap}(T).$$
\end{theorem}
\begin{proof}
With the previous lemma proven, the result follows almost immediately: for a normal operator we have

$$\|Tu\|^2 = (Tu, Tu) = (T^*Tu, u) = (TT^*u, u) = (T^*u, T^*u) = \|T^*u\|^2$$
and so $\|(T - \lambda)u_n\| = \|(T^* - \overline{\lambda})u_n\|$,
which means that $\Spec_{ap}(T) = \overline{\Spec_{ap} (T^*)}$. 
\end{proof}

This is fortunate, as many operators relevant to physical applications are
normal (if not self-adjoint); we can give a constructive definition for the
entire spectrum of a normal operator! 

We now specialise to a subset of the approximate point spectrum, known as the
essential spectrum. The essential spectrum has several definitions, 
the most popular usually denoted $\Spec_{e,i}$ for $i \in \{1,2,3,4,5\}$ in order of size. For 
most well-behaved operators the definitions are equivalent. This particular definition 
is known as Weyl's criterion, $\Spec_{e, 2}$ \cite{edmunds2018spectral}.

\begin{definition}[\textbf{Essential spectrum}]\index{essential spectrum}\index{Weyl's criterion}
  The essential spectrum of an operator $T$ on a Hilbert space $\hilbert$ is
  defined as the set of all $\lambda \in \mathbb{C}$ such that a
  \textbf{Weyl sequence}\index{Weyl sequence} $u_n$ exists for $T$ and
  $\lambda$ , i.e. a sequence with the properties:
  \begin{itemize}
    \item $\|u_n\| = 1\quad \forall n \in \mathbb{N}$;
    \item $u_n \rightharpoonup 0$ (where $\rightharpoonup$ denotes weak convergence:
	  $u_n \rightharpoonup u \Leftrightarrow (u_n, g) \rightarrow (u, g) \quad \forall g \in \hilbert$);
    \item $\lim_{n \rightarrow \infty}\|(T - \lambda)u_n\|  = 0$.
  \end{itemize}
\end{definition}

We can loosen the definition of weak convergence to just require convergence in
a dense subspace of $\hilbert$:
\begin{lemma}
\label{thm:weak-conv-dense-subset}
  A bounded sequence $u_n$, $\|u_n\| \leq C$, is weakly convergent to $u$ in
  $\hilbert$ if and only if it is weakly convergent to $u$ in $L$ where $L$ is a
  dense subspace of $\hilbert$. \end{lemma}
\begin{proof}
Weak convergence in $\hilbert$ implying the same in $L$ is obvious by the
definition.
Conversely, take $g \in \hilbert$. For any $\varepsilon > 0$, we have 
$\|g - \varphi\| < \varepsilon$ for $\varphi \in L$;
furthermore by the weak convergence of $u_n$ in $L$ we have $N \in
\mathbb{N}$ such that $(u_n - u, \varphi) < \varepsilon$ for $n \geq N$.
Then:
$$( u_n - u, g ) = ( u_n - u, g - \varphi + \varphi ) 
		 = ( u_n - u, g - \varphi ) + ( u_n - u, \varphi ) 
		 < \|u_n - u\| \|g - \varphi\| + \varepsilon < \varepsilon(C + 1) 
		 \rightarrow 0.$$
\end{proof}

We will now construct a Weyl sequence for the essential spectrum of the
'Laplacian' or 'free Schr\"odinger' operator $T = -\Delta$ on $L^2(\mathbb{R})$,
where on $\mathbb{R}^1$, $\Delta$ is the operator $\Delta f (x) =
\frac{d^2}{dx^2} f(x)$

\begin{example}
  The essential spectrum of the operator $T = -\Delta$ is the closed half-axis
  $[0, +\infty)$. 
\end{example}
\begin{proof}
First, note that for the exponential function we have 
\begin{equation}\label{eqn:laplace-eigenvector}
T\exp(i\omega x) = \omega^2 \exp(i\omega x).
\end{equation}
This gives much of the intuition for this proof; the function $\exp_{i \omega}:
x \mapsto \exp(i\omega x)$ is not an eigenvector as it is not in
$L^2(\mathbb{R})$, but it satisifes the eigenvalue equation for $T$ and
so any number $\lambda = \omega^2$ -
and thus any $\lambda \in [0, +\infty)$ - is 'almost' an eigenvalue for $T$.

We take advantage of this by choosing some smooth bump function $\rho \in
C^\infty_c(\mathbb{R})$ with $\|\rho\|_2 = 1$. We then define $\rho_n =
\frac{1}{\sqrt{n}}\rho(x/n)$. $\rho_n$ has some nice properties: by a
substitution of variables and direct calculation we have $\|\rho_n\|_2 =
\|\rho\|_2$, and furthermore any k'th derivative $\rho_n^{(k)}$ of
$\rho_n$ 
converges to 0 in $L^2$. Indeed:
\begin{equation}
\label{eqn:rhokn-vanishes}
\|\rho_n^{(k)}\|_2 = \frac{1}{n^k}\|\frac{1}{\sqrt{n}}\rho^{(k)}(x/n)\|_2 
		   = \frac{\|\rho^{(k)}\|_2}{n^k} 
		   \rightarrow 0
\end{equation}
where one can see $\|\frac{1}{\sqrt{n}}\rho^{(k)}(x/n)\|_2 = \|\rho^{(k)}\|_2$
by the same calculation as  $\|\rho_n\|_2 = \|\rho\|_2$.

Now, let our candidate Weyl sequence be $u_n: x \mapsto \rho_n(x)\exp(i\omega x)$,
which truncates $\exp(i\omega x)$ to $\supp \rho_n$; this means $u_n$ is in
$L^2(\mathbb{R})$. $\|u_n\| = \|\rho_n\|_2 = \|\rho\|_2 = 1$ by direct
calculation, and $u_n \rightharpoonup 0$: we can bound $u_n$ by
$\frac{1}{\sqrt{n}} M \1_{(\supp u_n)}$, where $\1_A$ is the
characteristic function of the set $A$ and $M$ is the maximum value of
$\rho$. Then by Lemma \ref{thm:weak-conv-dense-subset}, we can simply
show weak convergence for any $\varphi \in C_0^\infty$, which is dense
in $L^2$:
\begin{align*}
( u_n, \varphi ) & = \int_{\mathbb{R}}u_n \varphi & \\
& \leq \int_{\mathbb{R}} \frac{1}{\sqrt{n}} M \1_{(\supp u_n)} \varphi & \\
& \leq \int_{\supp \varphi} \frac{1}{\sqrt{n}} M  \varphi & \\
& = \frac{M}{\sqrt{n}} \int_{\supp \varphi} \varphi \rightarrow 0, 
  & \text{as the integral of $\varphi$ is finite and independent of $n$.}
\end{align*}
Finally, we show that $\lim_{n \rightarrow \infty}\|(T - \lambda)u_n\|_2 \rightarrow 0$
for $\lambda = \omega^2$:

\begin{align*}
\|(T - \lambda)u_n\|_2 & = \|(T(\exp_{i \omega} \rho_n) - \omega^2(\exp_{i \omega} \rho_n)\|_2 & \\
& = \|(T(\exp_{i \omega} \rho_n) - T(\exp_{i \omega}) \rho_n\|_2 
  & \text{\emph{(by equation (\ref{eqn:laplace-eigenvector}))}} \\
& =\|\exp_{i \omega} T\rho_n - 2 \omega \exp_{i \omega} \frac{d}{dx}\rho_n\|_u 
  & \text{\emph{(by the product rule)}} \\
& = \|T\rho_n - 2 \omega \frac{d}{dx}\rho_n\|_2 
  & \text{\emph{(see $\|\exp_{i \omega} \phi\|_2 = \|\phi\|_2$ for any $\phi \in L^2$)}} \\
& \leq  \|-\frac{d^2}{dx^2}\rho_n\|_2 + 2\omega\|\frac{d}{dx}\rho_n\|_2 \rightarrow 0, &
\end{align*}

converging by equation (\ref{eqn:rhokn-vanishes}).
Thus $u_n$ forms a Weyl sequence for $T$ and $\lambda \in [0, +\infty)$, as
required. 
\end{proof}

We can use a similar idea for another example to find the essential spectrum of
the multiplication operator:

\begin{example}
  Consider $f \in C(0, 1) \cap L^\infty (0, 1)$. The essential spectrum of the
  operator $M_f$ on $L^2(0, 1)$, where $M_f u(x) = f(x)u(x)$, is the range of $f$.
\end{example}
\begin{proof}
Similar to before, our initial idea comes from an 'almost-eigenvector'. In this
case, if $\lambda$ is in the range of $f$ with $f(x_0) = \lambda$, we
see that $M_f \delta_{x_0} = \lambda \delta_{x_0}$, where $\delta_{x_0}$
is the Dirac delta centred at $x_0$. Again, $\delta_{x_0}$ is not an eigenfunction of
$M_f$ as it is not in the correct domain - this time, it isn't even
strictly a function (it is a distribution).

Now consider a Friedrichs mollifier $\rho$. This is a function in
$C^\infty_0(\mathbb{R})$ with the property that $\sqrt{n}\rho(ny)
\rightarrow \delta_0$ as $n \rightarrow \infty$; we renormalise it such
that $\|\rho\|_2 = 1$, and take the sequence 
$$u_n: x \mapsto 
\begin{cases}
  \sqrt{n}\rho(n(x-x_0)) & x \in (0, 1) \\
  0 & \text{otherwise}
\end{cases}
$$
thus $u_n$ "converges to $\delta_{x_0}$" in the sense of distributions. Note
that $\|u_n\|_2 = \|\rho\|_2 = 1$ for all $n$, and this sequence
converges weakly to 0: 
\begin{align*}
|( u_n, g )| & = \int_{\supp{u_n}}\sqrt{n}\rho(n(x-x_0))g(x) 
	       & \text{\emph{(for any $g \in L^2(0, 1)$)}} \\
& \leq \|u_n\|_2 \sqrt{\int_{\supp{u_n}} |g(x)|^2} 
  & \text{\emph{(by H\"older's inequality)}} \\
& = \sqrt{\int_{\supp{u_n}} |g(x)|^2} \rightarrow 0, 
  \quad \text{as $\supp(u_n)$ decreases to 0.} &
\end{align*}

Then we see $\|(M_f - \lambda)u_n\|_2$ converges to zero by similar reasoning:
\begin{align*}
\|(M_f - \lambda)u_n\|^2_2 & = \int_{\supp \rho_n} |(f(x) - f(x_0) \sqrt{n} \rho(n(x-x_0))|^2 
			     & \text{\emph{(using that $\lambda = f(x_0)$)}} \\
& = \|(f(x) - f(x_0))^2\|_{L^\infty (\supp \rho_n)} \|\rho_n^2\|_1 
  & \text{\emph{(by H\"older's inequality)}} \\
& = \sup_{x \in \supp \rho_n} \|(f(x) - f(x_0))^2\| \rightarrow 0 
  & \text{\emph{(note $\|\rho_n^2\|_{L^1} = \|\rho_n\|_2 = 1$)}} \\
\end{align*}
converging to zero as $\supp \rho_n$ shrinks around $x_0$ by the continuity of $f$.
\end{proof}
Compare this example to Theorem \ref{thm:mult-op-spec}, and see that the
\emph{entire spectrum} of a multiplication operator is essential spectrum; the
operator has no eigenvalues.

One interesting property of the essential spectrum that is not easily visible
from our earlier definition is that it is invariant under compact perturbations.
This is not the case for eigenvalues!

\begin{definition}[\textbf{Compact operators and rank}]\index{compact operator}\index{rank of an operator}
  An operator $K$ on a normed vector space $X$ is \emph{compact} if for every
  bounded sequence $(x_n)_{n \in \mathbb{N}}$ in $X$, the sequence
  $(Kx_n)_{n \in \mathbb{N}}$ has a convergent subsequence.

  The rank of an operator $T$, denoted $\rank T$, is the dimension of its range.
\end{definition}

Note in particular that if a bounded operator $T$ has finite rank, then $T$ is
compact; as its image is finite-dimensional and bounded, the Bolzano-Weierstrass
theorem holds for $(Tx_n)_{n \in \mathbb{N}}$.

Note also that any compact operator is necessarily bounded, as otherwise we
could choose a bounded sequence $(x_n)_{n \in \mathbb{N}}$ in $\hilbert$ such
that $\|Tx_n\| \rightarrow \infty$, and then it would not be possible for
$(Tx_n)_{n \rightarrow \infty}$ to have a bounded subsequence. 

\begin{theorem}
\label{thm:ess-spec-comp-ptb}
  Let $\lambda$ be in the essential spectrum of an operator $T$ on a Hilbert space
  $\hilbert$. Then

  \begin{equation}
  \label{eqn:weyl_spectrum}
    \lambda \in \bigcap_{K \in \mathcal{K}(\hilbert)} \Spec(T + K)
  \end{equation}

  where $\mathcal{K}(\hilbert)$ is the space of all compact linear operators on $\hilbert$.
  Moreover, we have equality if $T$ is self-adjoint and bounded.
  This is often called the Weyl spectrum, which we will denote $\Spec_{e,4}$.
\end{theorem}
\begin{proof}
First, let $\lambda \in \Spec_e(T)$ have the Weyl sequence $x_n$ for the operator $T$. Then 

$$\|(T+K-\lambda)x_n\| = \|(T - \lambda)x_n+Kx_n\| \leq \|(T - \lambda)x_n\| + \|Kx_n\|$$. 

And as $K$ is compact, $Kx_n$ has a convergent subsequence $Kx_{n_k}$, where
$x_{n_k} \rightharpoonup 0$ because $x_n \rightharpoonup 0$. Then
$Kx_{n_k}$ also weakly converges to 0: for any $\phi \in \hilbert$,
\begin{align*}
(Kx_{n_k}, \phi) = (x_{n_k}, K^* \phi) \rightarrow 0
\end{align*}
by weak convergence of $x_{n_k}$ to 0. The result then follows by the uniqueness
of weak limits (i.e. if a sequence strongly converges to
\emph{something}, it must be the same value that it weakly converges to!)

Thus $\|(T+K-\lambda)x_{n_k}\| \leq \|(T - \lambda)x_{n_k}\| + \|Kx_{n_k}\|
\rightarrow 0$, so $x_{n_k}$ is a Weyl sequence for $\lambda$ and $T+K$
for any compact operator $K$, so $\lambda \in \bigcap_{K \in
\mathcal{K}(\hilbert)} \Spec(T + K)$.
\end{proof}

If the operator is self-adjoint, this becomes an equality and thus an equivalent
definition of the essential spectrum \cite{edmunds2018spectral}!

\begin{remark}
As mentioned before, eigenvalues do not have this property: let $\lambda$ be an
eigenvalue of $T$ with eigenvector $u$, and $P$ the orthogonal
projection onto the space $\mathrm{span}\{u\}$ (so $\rank P = 1$). Then we
have that $\lambda$ is not an eigenvalue of $(T+P)$, as otherwise:

\begin{align*}
  (T+P)u & = \lambda u \\
  Tu + Pu & = \lambda u \\
  \lambda u + u & = \lambda u
  \rightarrow u & = 0.
\end{align*}

However, it is trivial to
verify that $\lambda + 1$ is an eigenvalue of $(T+P)$ with eigenvector
$u$!) This will become critical in
discussing a method of detecting spectral pollution known as
\emph{dissipative barrier methods}, which we will explore in section
\ref{sec:dissipative-barrier}.
\end{remark}

\subsection{Rayleigh quotients and numerical range}\label{sec:num-range}
The gateway to bounding the spectrum (and pollution) of an operator $T$ on a
Hilbert space lies in a function known as the \textbf{Rayleigh
quotient}\index{Rayleigh quotient}, $R_T: \Dom(T) \rightarrow \mathbb{C}$,
defined:
\begin{equation*}
  R_T: u \mapsto \frac{( Tu, u )}{(u, u)}
\end{equation*} 
or equivalently (by linearity) $R_T: u \mapsto (Tu, u)$ on the domain $\{u \in
\Dom(T) : \|u\| = 1\}$.
 
For operators where we want to weaken the domain (e.g. a differential operator)
it is suitable to replace $(Tu, u)$ with the relevant bilinear form
$\mathcal{A}[u, u].$

One of the most important applications of the Rayleigh quotient (which is proven
in Corollary \ref{thm:self-adj-num-range}) is the \emph{min-max principle}, which
states for a self-adjoint operator which is bounded below,

\begin{definition}[\textbf{Numerical range of an operator}]
\index{numerical range}
  Let $T$ be an operator on a Hilbert space. The
  numerical range $\Num(T)$ is defined $\Num(T) \eqdef \Ran(R_T)$.
\end{definition}

The numerical range has a variety of interesting properties which make them
useful for roughly approximating the location of spectra.

\begin{proposition}
\label{thm:num-range-props}
  The numerical range $\Num(T)$ of an operator $T$ has the following properties:
  \begin{enumerate}
    \item
    \label{item:num-in-R}
	$\Num(T) \in \mathbb{R}$ if $T$ is self-adjoint;
    \item
    \label{item:proj-num-range} 
	$\Num(T_\mathcal{L}) \subseteq \Num(T)$, where $T_\mathcal{L}$ is the compression of $T$ to the closed subspace $\mathcal{L}$;
    \item (Toeplitz-Hausdorff theorem)
    \label{item:toeplitz-hausdorff}
      $\Num(T)$ is a convex set;
    \item
    \label{item:spec-in-num} 
      $\Spec_{ap}(T) \subseteq \overline{\Num(T)}$,
     where $\overline{\Num(T)}$ is the closure of the numerical range of $T$.
  \end{enumerate}
\end{proposition}

It is important to state the usefulness of these properties for spectral
pollution. Not only does $\Num(T)$ bound the spectrum of $T$, it bounds the
spectrum of $T_\mathcal{L}$ - effectively, bounding the region in which spectral
pollution can occur to a convex set around $\Spec(T)$.
We will also use this fourth property to derive a theorem on how well the
Galerkin method approximates the spectrum outside of $\conv(\Spec_{ess})$.

\begin{proof}
(\ref{item:num-in-R}.) If $T$ is self-adjoint, 
$(Tu, u) = (u, Tu)$ for all $u$;
by conjugate symmetry of scalar products, 
$(u, Tu) = \overline{(Tu, u)}$;
we combine these to find 
$(Tu, u) = \overline{(Tu, u)}$
and the result follows. 

(\ref{item:proj-num-range}.) 
$\Num(T_\mathcal{L}) = \{( PTPu, u ) : u \in \mathcal{L}, \|u\|=1\}$.
We then use the self-adjointness of $P$ to see
$$( PTPu, u ) = ( TPu, Pu ).$$
Then as $u \in \mathcal{L}, \|Pu\| = \|u\| = 1$,
so $( T(Pu), (Pu) ) \in \Num(T)$, and the result follows.

(\ref{item:toeplitz-hausdorff}. 
\cite{gustafson1997numerical})
Take $\lambda = (Tx, x), \mu = (Ty, y) \in \Num(T)$. Define the line segment
between them as $\nu = t\lambda + (1-t)\mu$ for $t \in [0, 1]$, and
$T_\mathcal{L}$ the compression of $T$ to the subspace 
$\mathcal{L} = \text{span}\{x, y\}$.
Then we note that $(T_\mathcal{L} x, x) = (Tx, x)$ and $(T_\mathcal{L} y, y) =
(Ty, y)$, so $\lambda, \mu$ are in $\Num(T_\mathcal{L})$.
$T_\mathcal{L}$ is two-dimensional, so is a $2 \times 2$ matrix; it can
be proven by direct calculation (see \cite{gustafson1997numerical}) that
the numerical range of a $2 \times 2$ matrix is an ellipse (with foci at
either eigenvalue of the matrix!) and so $\nu$ is in
$\Num(T_\mathcal{L})$.
Then by property \ref{item:proj-num-range}, $\Num(T_\mathcal{L}) \subseteq
\Num(T)$, so $\nu$ is also in $\Num(T)$, as required.

(\ref{item:spec-in-num}.) $\overline{\Num(T)}$ is the set of all points $\eta$
such that there is a sequence of unit vectors $u_n$ where
$$\lim_{n\rightarrow \infty}( Tu_n, u_n ) = \eta.$$

Now let $\lambda \in \Spec_{ap}(T)$. We can combine the approximate eigenvalue
theorem (Theorem \ref{thm:approx-eigenvalue-thm}) with the
Cauchy-Schwarz inequality: 
$$|( (T - \lambda)u_n, u_n )| \leq \|(T - \lambda)u_n\| \rightarrow 0,\text{ and so}$$
\begin{equation*}
\begin{split}
|( (T - \lambda)u_n, u_n )| &  = |(Tu_n, u_n) - ( \lambda u_n, u_n )| \\
& = |(Tu_n, u_n) - \lambda \|u_n\|^2| \\
& = |(Tu_n, u_n) - \lambda| \rightarrow 0; \\
& \Rightarrow ( Tu_n, u_n ) \rightarrow \lambda.
\end{split}
\end{equation*}
So $\lambda \in \overline{\Num(T)}$.
\end{proof}

\begin{corollary}\label{thm:normal-spec-in-num}
  If $T$ is a normal operator (i.e. $TT^* = T^*T$),
  then its entire spectrum is in the numerical range.
\end{corollary}
\begin{proof}
Combine Theorem \ref{thm:normal-spec} with Theorem \ref{thm:num-range-props}.\ref{item:spec-in-num}.
\end{proof}

\begin{theorem}\label{thm:self-adj-num-range}
  In particular, if $T$ is self-adjoint, $\Spec(T) \subseteq \mathbb{R}$.
  Furthermore, if $T$ is bounded, then we have 
  \begin{align*}
    \inf(\Spec(T)) & = \inf(\Num(T))\text{, and}\\
    \sup(\Spec(T)) & = \sup(\Num(T)).
  \end{align*}
\end{theorem}
\begin{proof}
We can see immediately that any self-adjoint operator is normal ($T = T^*
\Rightarrow T^*T = TT = TT^*$); thus the entire spectrum of the
self-adjoint operator is in the numerical range, and by Proposition
\ref{thm:num-range-props}.\ref{item:num-in-R} we have $\Spec(T)
\subseteq \overline{\Num(T)} \subseteq \mathbb{R}$.

Now, $\inf(\Spec(T)) \geq \inf(\Num(T))$. Let $\inf(\Num(T)) = w_0$; then for
any unit vector $u$, $((T - w_0)u, u)  = (Tu, u) - w_0 \geq 0$, and so
$u, v \mapsto ((T - w_0)u, v)$ defines a positive-semidefinite Hermitian
form, for which the Cauchy-Schwarz inequality holds\footnote{This can
easily be verified by looking at a standard proof for the inequality.}.
We then find the following bound for unit vectors $u, v$:
\begin{align*}
|\tau[u,v]|^2 \leq \tau[u, u] \tau[v, v] = ((T - w_0)u, u)((T - w_0)v, v) 
					 \leq \|T - w_0\|((T - w_0)u, u).
\end{align*}
Now let us take a minimising sequence $u_n$, $\|u_n\| = 1$ such that 
$(Tu_n, u_n) \rightarrow w_0$.
Then we have
\begin{equation*}
\|(T - w_0)u_n\|^2 = |\tau[u_n, (T - w_0)u_n]|^2  
		   \leq \|T - w_0\|((T - w_0)u_n, u_n) 
		   \leq \|T - w_0\||(Tu_n, u_n) - w_0| 
		   \rightarrow 0,
\end{equation*}
and therefore $w_0 \in \Spec(T)$ by Theorem \ref{thm:approx-eigenvalue-thm}.
The proof for the supremum $w_1$ is analogous with some sign changes.
\end{proof}
This corollary extends to self-adjoint
\textbf{semibounded}\index{semibounded operator} operators, which are operators
such that their Rayleigh quotient is bounded above or below by some constant $c$
- the result holds for the supremum or infimum for operators which are
semibounded above or below respectively. We omit this more general proof as we
would require a significant tangent to acquire the prerequisite results: see
e.g. (\cite{frank2022schrodinger}, Corollary 1.11). Moreover, note that a
bounded operator is semibounded both above and below.

\begin{corollary}\label{thm:poll-bound-num}
  Let $T$ be a self-adjoint, semibounded operator.
  Then spectral pollution does not occur outside of $\conv\Spec(T)$.
\end{corollary}
\begin{proof}
Combine Corollary \ref{thm:self-adj-num-range} with Proposition
\ref{thm:num-range-props}.\ref{item:proj-num-range}. 
\end{proof}

\begin{remark}
Finally, we will see an application of the Rayleigh quotient: we can use it
to create variational principles on the eigenvalues of certain operators.
  Let $T$ be a compact, self-adjoint operator on a Hilbert space $\hilbert$
  (which only has eigenvalues as it is
  compact), and list its eigenvalues in ascending order $\lambda_1 \leq \lambda_2 \leq \hdots$.
  The smallest eigenvalue $\lambda_1$ can be found as \cite{evans2010partial}:
$$\lambda_1 = \min_{u \in \hilbert} R_T(u).$$
\begin{proof}
  We see that by Theorem \ref{thm:self-adj-num-range},
  $\lambda_1 \leq R_T(u)$; and of course, for the normalised eigenfunction
  $\phi_1$ corresponding to $\lambda_1$, we have
  $$R_T(\phi_1) = (T\phi_1, \phi_1) = \lambda_1(\phi_1, \phi_1) = \lambda_1.$$
\end{proof}

  Moreover, we can obtain the \emph{min-max principle} \cite{pryce1993numerical}:
  if $S_k$ is any $k$-dimensional subspace of $\hilbert$,
  $$\lambda_k = \min_{S_k} \max_{u \in S_k} R_T(u).$$
\end{remark}

Is there a better bound than the one in Corollary \ref{thm:poll-bound-num}? 
From heuristic evidence we may expect one - if we look
at Example \ref{ex:schrodinger-ritz}, we can see that there is no pollution even
within the negative semiaxis, even though the region plotted is well within
$\conv\Spec(T)$. The numerical range has been refined in a variety
of ways, one of which is particularly profitable when it comes to bounding
spectral pollution. This shall be our next topic.

\subsection{Essential numerical range}
\index{essential numerical range}
A similar notion to that of the numerical range is the essential numerical
range, $\Num_e (T)$. This set lowers its aim to simply estimating the essential
spectrum, but in the process manages to do so much more accurately for some
operators.

\begin{definition}[\textbf{Essential numerical range}]
  (adapted from \cite{fillmore1972essential})
  The essential numerical range of an operator $T$ is given by\footnote{Much like
  the essential spectrum, there are multiple definitions of the essential
  numerical range. However (at least for bounded operators) there is much
  more equivalence between the definitions than we have for essential
  spectrum! \cite{fillmore1972essential} We choose the definition with the
  most natural relation to our choice of definition for essential spectrum.}

$$\Num_e (T) := \{\lim_{n \rightarrow \infty}( Tu_n, u_n ) 
		  : (u_n)_{n \in \mathbb{N}}\text{ in } \Dom(T), \|u_n\|=1,
		    u_n \rightharpoonup 0.\}$$
\end{definition}
Note the parallels with our definition of the essential spectrum, $\Spec_e(T)$.
Indeed, these parallels are reflected in the properties of $\Num_e(T)$:

\begin{proposition}
\label{thm:nume-props}
  The essential numerical range $\Num_e(T)$ of an operator $T$ has the following properties:
  \begin{enumerate}
    \item
    \label{item:nume-convex} 
      $\Num_e(T)$ is convex;
    \item
    \label{item:nume-in-clos-num} 
      $\Num_e(T) \subseteq \overline{\Num(T)}$;
    \item
    \label{item:nume-is-hull} 
      $\conv(\Spec_e(T)) \subseteq \Num_e(T)$, with equality if $T$ is self-adjoint and bounded.
\end{enumerate}
\end{proposition}

\begin{proof}
(\ref{item:nume-convex}. \cite{bogli2020essential}) We prove this by applying
the Toeplitz-Hausdorff theorem (Proposition
\ref{thm:num-range-props}.\ref{item:toeplitz-hausdorff}) to a sequence.
We take $\lambda = \lim_{n \rightarrow \infty}(Tx_n, x_n), \mu = \lim_{n
\rightarrow \infty}(Ty_n, y_n) \in \Num_e(T)$ and define a sequence
$$\nu_n = t(Tx_n, x_n) + (1-t)(Ty_n, y_n) \quad t \in [0, 1],$$ which
obviously converges to $\nu = t\lambda + (1-t)\mu$ as 
$n \rightarrow \infty$.
We then create a sequence of compressions $T_n$, where each compression is to
$\text{span}\{x_n, y_n\}$. By the Toeplitz-Hausdorff theorem, we know
that $\nu_n$ is in $\Spec(T_n)$ and get a sequence $\nu_n = (Tz_n, z_n)$
converging to $\nu$; the elements $z_n$ are unit vectors in
$\text{span}\{x_n, y_n\}$, and they weakly converge to 0 because $x_n$
and $y_n$ both do:
$$(z_n, g) = (\alpha x_n + \beta y_n, g) = \alpha(x_n, g) + \beta(y_n, g) \rightarrow 0 \quad \forall g \in \hilbert.$$
This means that $\nu = \lim_{n \rightarrow \infty}(Tz_n, z_n)$ is in $\Spec_e(T)$ as required.

(\ref{item:nume-in-clos-num}.) This can be seen directly from looking at the two
definitions. By definition, we have
  $$\overline{\Num(T)} = \{lim_{n \rightarrow \infty}( Tu_n, u_n ) 
			   : (u_n)_{n \in \mathbb{N}}\text{ in } \Dom(T),
			   \|u_n\|=1\},$$
and $\Num_e(T)$ is the subset of this with the extra condition
that $u_n \rightharpoonup 0$.

(\ref{item:nume-is-hull}.) The inclusion $\Spec_e(T) \subseteq \Num_e(T)$ comes
from an analogous argument to that of Proposition
\ref{thm:num-range-props}.\ref{item:spec-in-num}; then
$\conv(\Spec_e(T)) \subseteq \Num_e(T)$
by this inclusion and that $\Num_e(T)$ is a convex set.
\end{proof}
It remains to show that $\conv(\Spec_e(T)) = \Num_e(T)$ when $T$ is self-adjoint
and bounded. This will be proven after the following theorem, which describes
another similarity with essential spectra; invariance under compact
perturbation.

\begin{theorem}
\label{thm:ess-ran-compact-ptb}
  A value $\lambda$ is in the essential numerical range of an operator $T$ on a
  Hilbert space $\hilbert$ if
    $$\lambda \in \bigcap_{K \in \mathcal{K}(\hilbert)} \overline{\Num(T+K)}$$
  where $\mathcal{K}(\hilbert)$ is the set of all compact linear operators on $\hilbert$.
  We will use the standard notation
  $\Num_{e,3}(T) \eqdef \bigcap_{K \in \mathcal{K}(\hilbert)} \overline{\Num(T+K)}$. 
\end{theorem}
\begin{proof}
Let $\lambda = \lim_{n \rightarrow \infty}(Tx_n, x_n)$ be in $\Num_e(T)$. Then
let $K$ be a compact operator, and $x_{n_k}$ a subsequence such that $Kx_{n_k}$
is convergent (which exists as $K$ is compact). Let $Kx_{n_k} \rightarrow f \in \hilbert$. Then
  $$|(Kx_{n_k}, x_{n_k})| = |(Kx_{n_k} - f + f, x_{n_k})| 
    \leq \|Kx_{n_k} - f\|\|x_{n_k}\| + |(f, x_{n_k})| \rightarrow 0$$
by the weak convergence of $x_n$. Then $((T+K)x_{n_k}, x_{n_k}) \rightarrow \lambda$, as
$(Tx_{n_k}, x_{n_k}) \rightarrow \lambda$ by uniqueness of limits and 
$(Kx_{n_k}, x_{n_k}) \rightarrow 0$. This means that
$\lambda \in \Num_e(T+K) \subseteq \overline{\Num(T+K)}$ for all $K$, so $\lambda \in \Num_{e,3}(T).$
\end{proof}
The converse is also generally true; the proof of this involves considering the geometry of
$Num_e(T)$ and how it relates to the geometry of $\overline{\Num(T)}$ \cite{bogli2020essential}.
The following corollary will also show that they are equal for self-adjoint, bounded operators.

\begin{corollary}
  Let $T$ be self-adjoint and bounded. Then $\conv(\Spec_e(T)) = \Num_e(T)$.
\end{corollary}
\begin{proof}\emph{(sketch)}
We know that for any compact operator $K$,
  \begin{equation*}
  \tag{$\star$}
  \label{eqn:inclusions}
  \conv(\Spec_e(T)) \subseteq \Num_e(T) \subseteq \Num_{e,3}(T) \subseteq \overline{\Num(T+K)}.
  \end{equation*}
We then require an extension of one of our results, and an additional theorem:
  \begin{enumerate}
    \item\label{item:halmos} $\conv(\Spec(T)) = \Num(T)$ for any \emph{normal} bounded operator \cite{halmos1982hilbert};
    \item\label{item:salinas} If $T$ is \emph{hyponormal} (a generalisation of normal operators) then there is some 
      \emph{normal} compact $K$ such that $\Spec(T+K) = \Spec_{e,4}(T)$ \cite{salinas1972operators}.
  \end{enumerate}

Then for this normal compact $K$, we have by (\ref{item:halmos}) that
  $$\overline{\Num(T+K)} = \conv(\Spec(T+K))$$
  and by (\ref{item:salinas}), $\conv(\Spec(T+K)) = \conv(\Spec_{e,4}(T))$.

  Finally, we have that for self-adjoint $T$, $\Spec_{e,4}(T) = \Spec_e(T)$
  (Theorem \ref{thm:ess-spec-comp-ptb}), and
  thus $\overline{\Num(T+K)} \subseteq \conv(\Spec_e(T))$
  so the inclusions ($\star$) become equalities, as required.
\end{proof}

We have seen that the essential numerical range estimates the bounds of the
essential spectrum with quite astounding accuracy for some types of operator.
But the essential numerical range far outdoes the regular numerical range on
bounding spectral pollution; in fact, it provides the \emph{smallest} set on which
it is possible for pollution to occur! 

We will require the following prerequisite lemmata:
\begin{lemma}\label{thm:adjoint-spec-props}
  Let $T$ be a bounded operator on a Hilbert space $\hilbert$. Then:
  \begin{enumerate}
  \item
  \label{part:spec-of-adj}
    $\lambda$ is in $\Spec(T)$ if and only if $\overline{\lambda}$ is in $\Spec(T^*)$, and
  \item
  \label{part:adj-convergence}
    if $T_n$ is a sequence of truncations converging strongly to $T$, i.e. $T_n f
    \rightarrow Tf$ for any fixed $f \in \hilbert$, then $T_n^*$ converges
    strongly to $T^*$.
  \end{enumerate}
\end{lemma}
\begin{proof}
(\ref{part:spec-of-adj}.) Let $\mu$ be in the resolvent $\rho(T)$,
i.e. $(T - \rho)$ is invertible. Then we have
\begin{align*}
(T - \mu)(T - \mu)^{-1} & = I \\
((T - \mu)(T - \mu)^{-1})^* & = I^* = I \\
((T - \mu)^{-1})^*(T - \mu)^* & = I
\end{align*}
and similarly for $(T - \mu)^{-1}(T - \mu)$. Thus we have that 
$((T - \mu)^{-1})^*$ is the inverse of $(T - \mu)^* = (T^* - \overline{\mu})$,
so $\overline{\mu} \in \rho(T^*)$. We can repeat this argument for 
$(T^* - \overline{\mu})$, using that $T^{**} = T$. Thus $\mu \in \rho(T)
\Leftrightarrow \overline{\mu} \in \rho(T^*)$, and the result follows.

(\ref{part:adj-convergence}.) Let $T_n$ converge strongly to $T$. Then we have
  \begin{align*}
    \|T_n^*f\| = \|(P_n T)^* f\| & = \|T^* P_n^* f\| & \text{\emph{by antidistributivity}} \\
    & = \|T^* P_n f\| & \text{\emph{as projections are self-adjoint}} \\
    & \rightarrow \|T^* f\| & \text{\emph{when $n \rightarrow \infty$, $P_n \rightarrow I$}}.
  \end{align*}
\end{proof}

\begin{theorem}[\textbf{Pokrzywa \cite{pokrzywa1979method}}]
\label{thm:pokrzywa}
  Let $T$ be a bounded operator. All spectral pollution in the Ritz approximation
  of $\Spec(T)$ will be located inside of $\Num_e(T)$; within this set, it can
  occur anywhere in $\Num_e(T) \setminus \Spec(T)$.
\end{theorem}

\begin{proof}
Let the sequence of approximating operators be denoted $T_n = P_n
T\big|_{Ran(P_n)}$ for some projections $P_n$ (note that $P_n$ does not
necessarily have to be a projection onto the first $n$ orthonormal basis
functions, as the notation might suggest; it merely needs to be
subspaces of increasing dimension!). If $\lambda \in \mathbb{C}$ is a
point of spectral pollution, then there is some sequence of eigenpairs
$(\lambda_n, f_n)$, with $f_n$ normalised so $\|f_n\| = 1$, such that:
\begin{itemize}
  \item $(T_n - \lambda_n)f_n =0 \quad \forall n \in \mathbb{N}$;
  \item $\lambda_n \rightarrow \lambda$;
  \item $(T - \lambda)$ has a bounded inverse.
\end{itemize}
We then have that $(T_n - \lambda)f_n \rightarrow 0$:

\begin{equation}
\label{eqn:trunc-conv}
(T_n - \lambda)f_n = (T_n - \lambda_n + \lambda_n - \lambda)f_n 
		   = (T_n -\lambda_n)f_n + (\lambda_n - \lambda)f_n 
		   \rightarrow 0. 
\end{equation}

Furthermore, as $(T - \lambda)$ has a bounded inverse, so does $(T^* -
\overline{\lambda})$ by Lemma
\ref{thm:adjoint-spec-props}.\ref{part:spec-of-adj}. Then for any $h \in
\hilbert$, there is some $g \in \hilbert$ such that $(T^* - \overline{\lambda})g = h$.
As a result, we have
\begin{align*}
(f_n, h) & = (f_n, (T^* - \overline{\lambda})g) \\
& = (f_n, (T^* - \overline{\lambda})g - (T_n^* - \overline{\lambda})g) + (f_n, (T_n^* - \overline{\lambda})g) \\
& = (f_n, T^*g - T_n^*g) + ((T_n - \lambda)f_n, g) \\
& \rightarrow 0
\end{align*}
by Equation \ref{eqn:trunc-conv} and Lemma
\ref{thm:adjoint-spec-props}.\ref{part:adj-convergence}. Thus 
$f_n \rightharpoonup 0$.

Finally, we need that $(T f_n, f_n) \rightarrow \lambda$:
\begin{align*}
\lambda_n = (T_n f_n, f_n) & = (P_n T\big|_{\Ran(P_n)} f_n, f_n)  \\
& = (P_n T f_n, f_n) = (T f_n, f_n)
\end{align*}
as $P_n$ is self-adjoint, and $P_n f_n = f_n$. Thus $(T f_n, f_n) = \lambda_n
\rightarrow \lambda$, and so $\lambda$ is in the essential range of $T$ as
required. \end{proof}

For a bounded operator, this is the main result of a paper by Pokrzywa
\parencite{pokrzywa1979method}. The main theorem of the paper has the corollary
that for $\lambda \notin \Num_e(T)$, we have $\lambda \in \Spec(T)$ iff
$\text{dist}(\lambda, \Spec(T_n)) \rightarrow 0$; that is, outside of the
essential numerical range, every point in the approximate spectrum $\Spec(T_n)$
converges to a point in the actual spectrum of $T$. This is followed by a lemma
which claims that for any sequence $(\lambda_n)_{n \in \mathbb{N}}$ in
the interior of $\Num_e(T)$, there is a sequence of orthogonal projections such
that $\lambda_{n-1} \in \Spec(T_n)$ - not only does all spectral pollution occur
inside this range, but for \emph{any point} in $\Num_e(T) \setminus \Spec(T)$,
spectral pollution occurs there in some approximation.

\end{document}
