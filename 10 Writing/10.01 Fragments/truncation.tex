\documentclass{article}

\usepackage[utf8]{inputenc}
\usepackage[T1]{fontenc}
\usepackage{graphicx}
\usepackage[margin=0.6in]{geometry}
\usepackage{amsmath}
\usepackage{amssymb}

\usepackage[style=numeric, sorting=none]{biblatex}
\addbibresource{bibliography.bib}

\usepackage{hyperref}
\hypersetup{
    colorlinks=true,
    linkcolor=black,
    urlcolor=blue,
    citecolor=black
}

\usepackage{amsthm}
\newtheorem*{definition}{Definition}
\newtheorem{example}{Example}

\newcommand{\eqdef}{\mathrel{\mathop:}=}

\title{Truncation of an Operator}

\begin{document}
\maketitle

\begin{definition}{(Compression of an operator)}
Let $T$ be an operator on a Hilbert space $H$. The compression of
the operator to the closed linear subspace $\mathcal{L} \subseteq Dom(T)$,
denoted $T_\mathcal{L}$, is defined
$$T_\mathcal{L} \eqdef P_\mathcal{L} T\big|_{\mathcal{L}}$$
where $P_\mathcal{L}$ is the orthogonal projection of $H$ onto $\mathcal{L}$.
\end{definition}

This compression is also known as the \emph{truncation} of an operator. An equivalent (abuse of) notation 
is $PTP: \mathcal{L} \rightarrow \mathcal{L}$ (to imply the operator $PTP: H \rightarrow \mathcal{L}$ 
restricted to $\mathcal{L}$), however there can be confusion with the non-restricted operator 
(especially when we consider that if we are restricted to $\mathcal{L}$, the rightmost $P$ does not do anything!)



\begin{definition}{(Ritz approximation)}

\end{definition}

\printbibliography
\end{document}