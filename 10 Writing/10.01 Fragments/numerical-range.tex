\documentclass{article}

\usepackage[utf8]{inputenc}
\usepackage[T1]{fontenc}
\usepackage{graphicx}
\usepackage[margin=0.6in]{geometry}

\usepackage{amsmath}
\usepackage{amsthm}
\usepackage{amssymb}

\usepackage{enumitem}

\usepackage[style=numeric, sorting=none]{biblatex}
\addbibresource{bibliography.bib}

\usepackage{hyperref}
\hypersetup{
    colorlinks=true,
    linkcolor=black,
    urlcolor=blue,
    citecolor=black
}

\newcommand{\Num}{\text{Num}}
\newcommand{\Spec}{\text{Spec}}

\newtheorem{definition}{Definition}
\newtheorem{proposition}{Proposition}

\title{Numerical Range and Computing Spectra}

\begin{document}
\maketitle

\begin{definition}{(Numerical range of an operator)} Let $T$ be an operator on a Hilbert space $H$. The
numerical range $\Num(T)$ is defined

$$\Num(T) = \{\langle Tu, u \rangle : u \in \text{Dom}(T), \|u\|=1\}$$

where $\text{Dom}(T)$ is the domain of $T$.
\end{definition}

The numerical range has a variety of interesting properties which make them useful for roughly approximating
the location of spectra.

\begin{proposition}
The numerical range $\Num(T)$ of an operator $T$ has the following properties:
\begin{enumerate}
\item (Hausdorff–Toeplitz theorem) $\Num(T)$ is a convex set;
\item $\Num(T) \in \mathbb{R}$ iff $T$ is self-adjoint;
\end{enumerate}
And, most importantly for our discussion:
\begin{enumerate}[resume]
\item $\Spec(T) \subseteq \overline{\Num(T)}$, where $\overline{\Num(T)}$ is the closure of the numerical range of $T$;
\item $\Num(T_\mathcal{L}) \subseteq \Num(T)$, where $T_\mathcal{L}$ is the truncation of $T$ to the closed subspace $\mathcal{L}$.
\end{enumerate}
\end{proposition}

We will not prove the first two of these. For a proof of the Hausdorff–Toeplitz theorem see \cite{gustafson1997numerical}; the general idea
is to use \emph{4.} to reduce to the case of a two-dimensional operator, and then prove directly that the numerical range of a two-dimensional
operator is eliptical (with foci at either eigenvalue of $T$!).

Before proving the other two, it is important to state the usefulness of these properties. Not only does $\Num(T)$ bound the spectrum of $T$,
it bounds the spectrum of $T_\mathcal{L}$ - effectively, bounding the region in which spectral pollution can occur to a convex set around $\Spec(T)$.

\begin{proof}
(3.) $\overline{\text{Num}(T)}$ is the set of all points $\lambda$ such that there is a sequence of unit vectors $u_n$ where
$$\lim_{n\rightarrow0}\langle Tu_n, u_n \rangle = \lambda.$$

The approximate point spectrum
$\sigma_{ap}(T) = \{\lambda \in \mathbb{C}: \text{lim}_{n \rightarrow \infty}\|(T - \lambda)u_n\| \text{ for some sequence } (u_n), \|u_n\| = 1\}$
can be shown to contain the boundary of the spectrum of $T$. Then by the Cauchy-Schwarz inequality,
$|\langle (T - \lambda)u_n, u_n \rangle| \leq \|(T - \lambda)u_n\| \rightarrow 0$, and so
\begin{equation*}
\begin{split}
|\langle (T - \lambda)u_n, u_n \rangle| &  = |\langle (Tu_n, u_n) \rangle - \langle \lambda u_n, u_n \rangle| \\
& = |\langle (Tu_n, u_n) \rangle - \lambda \|u_n\|^2| \\
& = |\langle (Tu_n, u_n) \rangle - \lambda| \rightarrow 0; \\
& \implies \langle Tu_n, u_n \rangle \rightarrow \lambda.
\end{split}
\end{equation*}

Thus the boundary of the spectrum is in the $\overline{\Num(T)}$; as the numerical range is convex,
this means the whole spectrum must be.\\

(4.) $\Num(T_\mathcal{L}) = \{\langle PTPu, u \rangle : u \in \mathcal{L}, \|u\|=1\}$.
We then use the self-adjointness of $P$ to see
$$\langle PTPu, u \rangle = \langle TPu, Pu \rangle$$
Then as $u \in \mathcal{L}, \|Pu\| = \|u\| = 1$,
so $\langle T(Pu), (Pu) \rangle \in \text{Num}(T)$, and the result follows.

\end{proof}

\printbibliography
\end{document}