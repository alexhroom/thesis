\documentclass{article}

\usepackage[utf8]{inputenc}
\usepackage[T1]{fontenc}
\usepackage{graphicx}
\usepackage[margin=0.6in]{geometry}

\usepackage{amsmath}
\usepackage{amsthm}
\usepackage{amssymb}

\usepackage{enumitem}

\usepackage[style=numeric, sorting=none]{biblatex}
\addbibresource{bibliography.bib}

\usepackage{hyperref}
\hypersetup{
    colorlinks=true,
    linkcolor=black,
    urlcolor=blue,
    citecolor=black
}

\newcommand{\Num}{\text{W}}
\newcommand{\Spec}{\text{Spec}}
\newcommand{\Dom}{\text{Dom}}
\newcommand{\conv}{\text{conv}}

\newtheorem{definition}{Definition}
\newtheorem{proposition}{Proposition}[section]
\newtheorem{theorem}[proposition]{Theorem}

\title{Numerical Range}

\begin{document}
\maketitle

\section{Numerical range}

\begin{definition}{\textbf{(Numerical range of an operator)}} Let $T$ be an operator on a Hilbert space $H$. The
numerical range $\Num(T)$ is defined

$$\Num(T) = \{( Tu, u ) : u \in \Dom(T), \|u\|=1\}$$

where $\text{Dom}(T)$ is the domain of $T$.
\end{definition}

The numerical range has a variety of interesting properties which make them useful for roughly approximating
the location of spectra.

\begin{proposition}\label{thm:num-range-props}
The numerical range $\Num(T)$ of an operator $T$ has the following properties:
\begin{enumerate}
\item (Hausdorff–Toeplitz theorem) $\Num(T)$ is a convex set;
\item\label{item:num-in-R} $\Num(T) \in \mathbb{R}$ iff $T$ is self-adjoint;
\end{enumerate}
And, most importantly for our discussion:
\begin{enumerate}[resume]
\item\label{item:spec-in-num} $\Spec(T) \subseteq \overline{\Num(T)}$, where $\overline{\Num(T)}$ is the closure of the numerical range of $T$;
\item\label{item:proj-num-range} $\Num(T_\mathcal{L}) \subseteq \Num(T)$, where $T_\mathcal{L}$ is the truncation of $T$ to the closed subspace $\mathcal{L}$.
\end{enumerate}
\end{proposition}

We will not prove the Toeplitz-Hausdorff theorem. For an elementary proof see \cite{gustafson1997numerical}; the general idea
is to use property (\ref{item:proj-num-range}.) to reduce to the case of a two-dimensional operator, and then prove directly that the numerical range of a two-dimensional
operator is eliptical (with foci at either eigenvalue of $T$!).

Before proving the other two, it is important to state the usefulness of these properties. Not only does $\Num(T)$ bound the spectrum of $T$,
it bounds the spectrum of $T_\mathcal{L}$ - effectively, bounding the region in which spectral pollution can occur to a convex set around $\Spec(T)$.

\begin{proof}
(\ref{item:num-in-R}.) If $T$ is self-adjoint, then $(Tu, u) = (u, Tu)$; by conjugate symmetry of scalar products, $(u, Tu) = \overline{(Tu, u)}$; therefore we combine these to find $(Tu, u) = \overline{(Tu, u)}$ and thus $(Tu, u)$ is real.

(\ref{item:spec-in-num}.) $\overline{\Num(T)}$ is the set of all points $\lambda$ such that there is a sequence of unit vectors $u_n$ where
$$\lim_{n\rightarrow0}( Tu_n, u_n ) = \lambda.$$

The approximate point spectrum
$\sigma_{ap}(T) = \{\lambda \in \mathbb{C}: \text{lim}_{n \rightarrow \infty}\|(T - \lambda)u_n\| \text{ for some sequence } (u_n)_{n \in \mathbb{N}}, \|u_n\| = 1\}$
can be shown to contain the boundary of the spectrum of $T$ (e.g. in \citetitle{halmos1982hilbert} \cite{halmos1982hilbert}, problem 78). Then by the Cauchy-Schwarz inequality,
$|( (T - \lambda)u_n, u_n )| \leq \|(T - \lambda)u_n\| \rightarrow 0$, and so
\begin{equation*}
\begin{split}
|( (T - \lambda)u_n, u_n )| &  = |( (Tu_n, u_n) ) - ( \lambda u_n, u_n )| \\
& = |( (Tu_n, u_n) ) - \lambda \|u_n\|^2| \\
& = |( (Tu_n, u_n) ) - \lambda| \rightarrow 0; \\
& \implies ( Tu_n, u_n ) \rightarrow \lambda.
\end{split}
\end{equation*}

Thus the boundary of the spectrum is in $\overline{\Num(T)}$; as the numerical range is convex,
this means the whole spectrum must be.\\

(\ref{item:proj-num-range}.) $\Num(T_\mathcal{L}) = \{( PTPu, u ) : u \in \mathcal{L}, \|u\|=1\}$.
We then use the self-adjointness of $P$ to see
$$( PTPu, u ) = ( TPu, Pu )$$
Then as $u \in \mathcal{L}, \|Pu\| = \|u\| = 1$,
so $( T(Pu), (Pu) ) \in \Num(T)$, and the result follows.

\end{proof}

\section{Essential numerical range}
A similar notion to that of the numerical range is the essential numerical range, $\Num_e (T)$. This set lowers its aim to simply estimating the essential spectrum,
but in the process manages to do so much more accurately for some operators.

\begin{definition}{\textbf{(Essential numerical range)}} (adapted from \cite{fillmore1972essential})
The essential numerical range of an operator $T$ is given by\footnote{Much like the essential spectrum, there are multiple definitions of the essential numerical range.
However, there is much more equivalence between the definitions than we have for essential spectrum! \cite{fillmore1972essential} We choose the definition with the most
natural relation to our choice of definition for essential spectrum.}

$$\Num_e (T) := \{\lim_{n \rightarrow \infty}( Tu_n, u_n ) : (u_n)_{n \in \mathbb{N}}\text{ in } \Dom(T), \|u_n\|=1, u_n \rightharpoonup 0.\}$$
\end{definition}
Note the parallels with our definition of the essential spectrum, $\Spec_e(T)$. Indeed, these parallels are reflected in the properties of $\Num_e(T)$:

\begin{proposition}\label{thm:nume-props} The essential numerical range $\Num_e(T)$ of a bounded operator $T$ has the following properties:
\begin{enumerate}
\item\label{item:nume-convex} $\Num_e(T)$ is convex;
\item\label{item:nume-in-clos-num} $\Num_e(T) \subseteq \overline{\Num(T)}$;
\item\label{item:nume-is-hull} $\conv(\Spec_e(T)) \subseteq \Num_e(T)$, with equality if $T$ is self-adjoint.
\end{enumerate}
\end{proposition}
(\ref{item:nume-convex}.) is proven analogously to the Toeplitz-Hausdorff theorem; that is, we take two points
$\lambda, \mu \in \Num_e(T)$ and reduce the problem to a two-dimensional problem in their span. \cite{bogli2020essential}

\begin{proof}
(\ref{item:nume-in-clos-num}.) This can be seen directly from looking at the two definitions. By definition, we have
$$\overline{\Num(T)} = \{( Tu_n, u_n ) : (u_n)_{n \in \mathbb{N}}\text{ in } \Dom(T), \|u_n\|=1\},$$ and $\Num_e(T)$ is the subset of this with the extra condition
that $u_n \rightharpoonup 0$.
\\
(\ref{item:nume-is-hull}.) The inclusion $\Spec_e(T) \subseteq \Num_e(T)$ comes from an analogous argument
to that of Proposition \ref{thm:num-range-props}.\ref{item:spec-in-num}; then $\conv(\Spec_e(T)) \subseteq \Num_e(T)$
by this inclusion and that $\Num_e(T)$ is a convex set. It remains to show that $\conv(\Spec_e(T)) = \Num_e(T)$ when $T$ is self-adjoint and bounded.
\end{proof}

It is possible to weaken Proposition \ref{thm:nume-props}.\ref{item:nume-is-hull} to include a wider set of operators; Salinas \parencite{salinassomethingsomething} showed that it is in fact true for any bounded \emph{hyponormal} operator, which is a bounded operator $A$ where $A^*A - AA^* \leq 1$ (every normal operator is hyponormal, and every self-adjoint operator is normal, from which we recover our statement).

We have seen that the essential numerical range estimates the bounds of the essential spectrum with quite astounding accuracy for some types of
operator. But the essential numerical range far outdoes the regular numerical range on bounding spectral pollution; in fact, it provides an \emph{exact}
set on which it is possible for pollution to occur! 

\begin{theorem}
All spectral pollution in the Ritz approximation of $\Spec(T)$ will be located inside of $\Num_e(T)$; within this set, it can occur anywhere in $\Num_e(T) \setminus \Spec(T)$.
\end{theorem}

For a bounded operator, this is the main result of a paper by Pokrzywa \parencite{pokrzywa1979method}. The main theorem of the paper has the
corollary that for $\lambda \notin \Num_e(T)$, we have $\lambda \in \Spec(T)$ iff $\text{dist}(\lambda, \Spec(T_n)) \rightarrow 0$; that is, outside of the
essential numerical range, every point in the approximate spectrum $\Spec(T_n)$ converges to a point in the actual spectrum of $T$. This is followed by
a lemma which claims that for any sequence $(\lambda_n)_{n \in \mathbb{N}}$ in the interior of $\Num_e(T)$, there is a sequence of orthogonal projections such that $\lambda_{n-1} \in \Spec(T_n)$ - not only does all spectral pollution occur inside this range, but for \emph{any point} in 
$\Num_e(T) \setminus \Spec(T)$, spectral pollution occurs there in some approximation.


\printbibliography
\end{document}