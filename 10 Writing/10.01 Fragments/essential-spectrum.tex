\documentclass{article}

\usepackage[utf8]{inputenc}
\usepackage[T1]{fontenc}
\usepackage{graphicx}
\usepackage[margin=1in]{geometry}
\usepackage{amsmath}
\usepackage{amssymb}
\usepackage{mathtools}

\usepackage[style=numeric, sorting=none]{biblatex}
\addbibresource{bibliography.bib}

\usepackage{hyperref}
\hypersetup{
    colorlinks=true,
    linkcolor=black,
    urlcolor=blue,
    citecolor=black
}

\usepackage{amsthm}
\newtheorem*{definition}{Definition}
\newtheorem*{remark}{Remark}
\newtheorem{proposition}{Proposition}
\newtheorem{lemma}{Lemma}

\newcommand{\Spec}{\text{Spec}}
\newcommand{\supp}{\text{supp}}
\newcommand{\1}{\mathbf{1}}

\title{Essential Spectrum of the Laplace and Multiplication operators}
\author{Alex H. Room}

\begin{document}
\maketitle

\begin{definition}{\textbf{(Essential spectrum)}}
The essential spectrum\footnote{The essential spectrum has several definitions, 
the most popular usually denoted $\Spec_{e,i}$ for $i \in \{1,2,3,4,5\}$ in order of size. For 
most well-behaved operators the definitions are equivalent. This particular definition 
is Weyl's criterion, $\Spec_{e, 2}$.} of an operator $T$ on a Hilbert space $H$ 
is defined as the set of all $\lambda$ such that a \textbf{Weyl sequence} $u_n$ exists for $T$ and $\lambda$ , i.e. a sequence with the properties:
\begin{itemize}
\item $\|u_n\| = 1\quad \forall n \in \mathbb{N}$;
\item $u_n \rightharpoonup 0$ (where $\rightharpoonup$ denotes weak convergence: $u_n \rightharpoonup u \Leftrightarrow (u_n, g) \rightarrow (u, g) \quad \forall g \in H$);
\item $\lim_{n \rightarrow \infty}\|(T - \lambda)u_n\|  \rightarrow 0$.
\end{itemize}
\end{definition}
We can loosen the definition of weak convergence to just require convergence in a dense subspace of $H$:
\begin{lemma}\label{thm:weak-conv-dense-subset}
A bounded sequence $u_n$, $\|u_n\| \leq C$, is weakly convergent to $u$ in $H$ if and only if it is weakly convergent to $u$ in $L$ where $L$ is a dense subspace of $H$.
\end{lemma}
\begin{proof}
Weak convergence in $H$ implying the same in $L$ is obvious by the definition. 
Conversely, take $g \in H$. For any $\varepsilon > 0$, we have $\|g - \varphi\| < \varepsilon$ for $\varphi \in L$; furthermore by the weak convergence of $u_n$ in $L$ we have $N \in \mathbb{N}$ such that $(u_n - u, \varphi) < \varepsilon$ for $n \geq N$. Then:
$$( u_n - u, g ) = ( u_n - u, g - \varphi + \varphi ) = ( u_n - u, g - \varphi ) + ( u_n - u, \varphi ) < \|u_n - u\| \|g - \varphi\| + \varepsilon < \varepsilon(C + 1) \rightarrow 0.$$
\end{proof}

We will now construct a Weyl sequence for the essential spectrum of the 'Laplacian' or 'free Schr\"odinger' operator $T = -\Delta$ on $L^2(\mathbb{R})$,
where $\Delta$ is the operator $\Delta f (x) = \frac{d^2}{dx^2} f(x)$

\begin{proposition}
The essential spectrum of the operator $T = -\Delta$ is the closed half-axis $[0, +\infty)$.
\end{proposition}
\begin{proof}
First, note that for the exponential function we have 
\begin{equation}\label{eqn:laplace-eigenvector}
T(\text{exp})(i\omega x) = \omega^2 \exp(i\omega x).
\end{equation}
This gives much of the intuition for this proof; the function $\exp_{i \omega}: x \mapsto \exp(i\omega x)$ is not an eigenvector 
as it is not in $L^2(\mathbb{R})$, but it satisifes the eigenvalue equation for $T$ and so any number $\lambda = \omega^2$ -
and thus any $\lambda \in [0, +\infty)$ - is 'almost' an eigenvalue for $T$.

We take advantage of this by choosing some smooth bump function $\rho \in C^\infty_c(\mathbb{R})$ with $\|\rho\|_2 = 1$. We
then define $\rho_n = \frac{1}{\sqrt{n}}\rho(x/n)$. $\rho_n$ has some nice properties: by a substitution of variables and direct calculation we have $\|\rho_n\|_2 = \|\rho\|_2$, and furthermore any k'th derivative $\rho_n^{(k)}$ of $\rho_n$ 
converges to 0 in $L^2$. Indeed:
\begin{equation}\label{eqn:rhokn-vanishes}
\|\rho_n^{(k)}\|_2 = \frac{1}{n^k}\|\frac{1}{\sqrt{n}}\rho^{(k)}(x/n)\|_2 = \frac{\|\rho^{(k)}\|_2}{n^k} \rightarrow 0
\end{equation}
where one can see $\|\frac{1}{\sqrt{n}}\rho^{(k)}(x/n)\|_2 = \|\rho^{(k)}\|_2$ by the same calculation as  $\|\rho_n\|_2 = \|\rho\|_2$.

Now, let our candidate Weyl sequence be $u_n: x \mapsto \rho_n(x)\exp(i\omega x)$,
which truncates $\exp(i\omega x)$ to $\supp \rho_n$; this means $u_n$ is in $L^2(\mathbb{R})$.
$\|u_n\| = \|\rho_n\|_2 = \|\rho\|_2 = 1$ by direct calculation, and $u_n \rightharpoonup 0$: we can bound $u_n$ by $\frac{1}{\sqrt{n}} M \1_{(\supp u_n)}$, where $\1_A$ is the characteristic function of the set $A$ and $M$ is the maximum value of $\rho$. Then by Lemma \ref{thm:weak-conv-dense-subset}, we can simply show weak convergence for any $\varphi \in C_0^\infty$, which is dense in $L^2$:
\begin{align*}
( u_n, \varphi ) & = \int_{\mathbb{R}}u_n \varphi & \\
& \leq \int_{\mathbb{R}} \frac{1}{\sqrt{n}} M \1_{(\supp u_n)} \varphi & \\
& \leq \int_{\supp \varphi} \frac{1}{\sqrt{n}} M  \varphi & \\
& = \frac{M}{\sqrt{n}} \int_{\supp \varphi} \varphi \rightarrow 0, & \text{as the integral of $\varphi$ is finite and independent of $n$.}
\end{align*}
Finally, we show that $\lim_{n \rightarrow \infty}\|(T - \lambda)u_n\|_2  \rightarrow 0$ for $\lambda = \omega^2$:

\begin{align*}
\|(T - \lambda)u_n\|_2 & = \|(T(\exp_{i \omega} \rho_n) - \omega^2(\exp_{i \omega} \rho_n)\|_2 & \\
& = \|(T(\exp_{i \omega} \rho_n) - T(\exp_{i \omega}) \rho_n\|_2 & \text{\emph{(by equation (\ref{eqn:laplace-eigenvector}))}} \\
& =\|\exp_{i \omega} T\rho_n - 2 \omega \exp_{i \omega} \frac{d}{dx}\rho_n\|_2 & \text{\emph{(by the product rule)}} \\
& = \|T\rho_n - 2 \omega \frac{d}{dx}\rho_n\|_2 & \text{\emph{(see $\|\exp_{i \omega} \phi\|_2 = \|\phi\|_2$ for any $\phi \in L^2$)}} \\
& \leq  \|-\frac{d^2}{dx^2}\rho_n\|_2 + 2\omega\|\frac{d}{dx}\rho_n\|_2 \rightarrow 0, & 
\end{align*}

converging by equation (\ref{eqn:rhokn-vanishes}).
Thus $u_n$ forms a Weyl singular sequence for $T$ and $\lambda \in [0, +\infty)$, as required.
\end{proof}

We can use a similar idea for another example to find the essential spectrum of the multiplication operator:

\begin{proposition}
The essential spectrum of the operator $M_f$ on $L^2(0, 1)$, where $M_f u(x) = f(x)u(x)$, is the range of $f$.
\end{proposition}
\begin{proof}
Similar to before, our initial idea comes from an 'almost-eigenvector'. In this case, if $\lambda$ is in the range
of $f$ with $f(x_0) = \lambda$, we see that $M_f \delta_{x_0} = \lambda \delta_{x_0}$, where $\delta_{x_0}$ is the
Dirac delta centred at $x_0$. Again, $\delta_{x_0}$ is not an eigenfunction of $M_f$ as it is not in the correct domain - this time, 
it isn't even strictly a function (it is a distribution).

Now consider a Friedrichs mollifier $\rho$. This is a function in $C^\infty_0(\mathbb{R})$ with the property that $\sqrt{n}\rho(ny) \rightarrow \delta_0$ as $n \rightarrow \infty$; we renormalise it such that $\|\rho\|_2 = 1$, and take the sequence 

$$u_n: x \mapsto 
\begin{cases}
  \sqrt{n}\rho(n(x-x_0)) & x \in (0, 1) \\
  0 & \text{otherwise}
\end{cases}
$$

thus $u_n$ "converges to $\delta_{x_0}$" in the sense of distributions. Note that $\|u_n\|_2 = \|\rho\|_2 = 1$ for all $n$, and this sequence converges weakly to 0:
\begin{align*}
|( u_n, g )| & = \int_{\supp{u_n}}\sqrt{n}\rho(n(x-x_0))g(x) & \text{\emph{(for any $g \in L^2(0, 1)$)}} \\
& \leq \|u_n\|_2 \sqrt{\int_{\supp{u_n}} |g(x)|^2} & \text{\emph{(by H\"older's inequality)}} \\
& = \sqrt{\int_{\supp{u_n}} |g(x)|^2} \rightarrow 0, \quad \text{as $\supp(u_n)$ decreases to 0.} &
\end{align*}

Then we see $\|(M_f - \lambda)u_n\|_2$ converges to zero by similar reasoning:
\begin{align*}
\|(M_f - \lambda)u_n\|^2_2 & = \int_{\supp \rho_n} |(f(x) - f(x_0) \sqrt{n} \rho(n(x-x_0))|^2 & \text{\emph{(using that $\lambda = f(x_0)$)}} \\
& = \|(f(x) - f(x_0))^2\|_{L^\infty (\supp \rho_n)} \|\rho_n^2\|_1 & \text{\emph{(by H\"older's inequality)}} \\
& = \sup_{x \in \supp \rho_n} \|(f(x) - f(x_0))^2\| \rightarrow 0 & \text{\emph{(note $\|\rho_n^2\|_{L^1} = \|\rho_n\|_2 = 1$)}} \\
\end{align*}
converging to zero as $\supp \rho_n$ shrinks around $x_0$ by the continuity of $f$.

\end{proof}

\printbibliography
\end{document}