\documentclass{article}

\usepackage[utf8]{inputenc}
\usepackage{graphicx}
\usepackage[margin=1in]{geometry}
\usepackage[justification=centering]{caption}
\usepackage{enumerate}
\usepackage{csquotes}

\usepackage[style=numeric, sorting=none]{biblatex}
\addbibresource{bibliography.bib}

\usepackage{utopia}
\renewcommand{\sfdefault}{cmr}

\usepackage[x11names]{xcolor}
\definecolor{IKB}{HTML}{002FA7}

\usepackage[explicit]{titlesec}
\titleformat{\section}[hang]{\newpage\color{IKB}\Large\scshape\bfseries}{\Roman{section}}{0.61803cm}{\rule[-0.25cm]{1.5pt}{1cm}\quad#1}

\usepackage{hyperref}
\hypersetup{
    colorlinks=true,
    linkcolor=black,
    urlcolor=blue,
    citecolor=black
}

\usepackage{amssymb}
\newcommand{\Spec}{\text{Spec}}
\newcommand{\1}{\mathbf{1}}

\usepackage{amsmath}
\usepackage{amsthm}
\newtheorem{definition}{Definition}[section]
\newtheorem{theorem}{Theorem}[section]
\newtheorem{lemma}[theorem]{Lemma}
\newtheorem{example}{Example}[section]

\begin{document}

\section{Spectral pollution in a multiplication operator}
\subsection{The spectrum of a multiplication operator}
\begin{definition}{\textbf{(Multiplication operator)}}
For a given function $a \in L^\infty(\Omega)$, the multiplication operator $M_a$ on $L^2(\Omega)$ is defined by
the action $M_af(x) = a(x)f(x)$; that is, it acts by pointwise multiplication with $a$. We call $a$ the `symbol' of
the multiplication operator.
\end{definition}

The spectrum of a multiplcation operator is easy to calculate. We first
need to define the 'essential range' of a function. Intuitively, this is
similar to the standard range of a function, but ignoring values taken by
the function on a set of measure zero - two functions which are equal
almost everywhere will have the same essential range.
The following classical results are taken from \cite{davies1995spectral}.

\begin{definition}{\textbf{(Essential range)}}\label{defn:essential-range}
  The essential range of a real-valued function $a$ is the set:
  $$\{k \in \mathbb{R} : \forall \varepsilon > 0, \mu\{x : |a(x) - k| < \varepsilon\} > 0\}$$
  where $\mu$ is the Lebesgue measure.
\end{definition}

\begin{theorem}\label{thm:mult-op-spec}
  The spectrum of the multiplication operator $M_a$ is the essential range of its symbol $a$.
\end{theorem}
\begin{proof}
  The spectrum of $M_a$ is the set of $\lambda$ for which $(A - \lambda I)$ has no inverse. Note
  for a function $f$, $(A - \lambda I)f(x) = (a(x) - \lambda)f(x).$.

  Let $\lambda$ be outside of the essential range of $a$. Then see the function
  $$g(x) = \frac{1}{\lambda - a(x)}$$
  is defined and bounded almost everywhere, so defines a multiplication operator $M_g$ on $L^2$.
  It is then easy to see that $M_g((a(x) - \lambda)f(x)) = f(x)$, meaning
  $M_g$ is an inverse to $(A - \lambda I)$, so $\lambda \notin \Spec(M_a)$.

  Conversely, take $\lambda$ in the essential range of $a$. Then we take the sets
  $S_n := {x: |a(x) - \lambda| < 2^{-n}}$. By definition of the essential range,
  these sets have non-zero measure. Then with $\phi_n$ = $\1_{S_n}$ as the indicator
  functions of $S_n$, we see
  $$\|M_a\phi_n - \lambda \phi_n\| < 2^{-n}\|\phi_n\|$$
  and so $\lambda \in \Spec(M_a)$.
\end{proof}

We will now take advantage of the ability to create simply-defined operators with easy-to-calculate
spectra to observe the existence of spectral pollution.

\subsubsection{Estimating spectra computationally}
\begin{example}
Let $M_f$ be the multiplication operator on $L^2(0, 1)$ with symbol

$$
f: x \mapsto \begin{cases}
x & x < 1/2 \\
x + 1/2 & \text{otherwise.}
\end{cases}
$$
By Theorem \ref{thm:mult-op-spec}, the spectrum of $M_f$ is the set $[0, 1/2] \cup [1, 3/2]$. If our Ritz approximation works as expected, we should
see the approximation create two dense clusters of eigenvalues which approach these intervals.
\end{example}

\subsubsection{The structure of approximations}
If we take a closer look at the structure of our approximating Ritz matrices we uncover deeper structure, which will lead
to our next topic.
\begin{example}\label{exp:mult-op-toeplitz}
Let $M_f$ be a multiplication operator on $L^2(0, 1)$. Now we create the Ritz matrix, $A_{j,k} = (M_f \phi_j, \phi_k)$, 
choosing the orthonormal basis $\phi_n(x) = \exp(2 \pi i n x).$

We now note that there is a structure to our matrix:
\begin{align*}
(M_f \phi_j, \phi_k) & = \int_0^1 f(x) \exp(2 \pi i j x) \exp(-2 \pi i k x) dx \\
& = \int_0^1 f(x) \exp(2 \pi i (j-k) x)\\
& = c_{j-k}
\end{align*}
where $c_n$ is the n'th Fourier coefficient. Thus our Ritz matrix depends only on the value of $j-k$; in particular, it is constant
along each diagonal. This is a special type of matrix known as a Toeplitz matrix.
\end{example}
As a result, the approximation of our operator $M_f$ is equal to the approximation of a infinite matrix $(T_f)_{j,k} = c_{j-k},
j,k \in \mathbb{N}_0$ by its truncations $(T_{f,n})_{j,k} = c_{j-k}$ for $j, k \leq n$. And nowhere in this derivation did we use particular
properties of $f$; we can repeat this reasoning with any function capable of being represented by a Fourier series. Let us systematise
what we have seen.

\subsection{Toeplitz operators}
\subsubsection{Toeplitz operators, Toeplitz matrices, and their equivalence}

The approximation of spectra provides a natural gateway to the study of Toeplitz operators, a type of operator with an elegant cluster
of representations that will provide intuitive and concrete insight into spectral pollution. A full account of Toeplitz theory is the subject
of whole monographs (such as \parencite{bottcher2006analysis}) and is an entire subfield in itself;
here we will stick to exploring their spectra, and how these relate to the spectra of their multiplication operator neighbours. 

\begin{definition}{\textbf{(Toeplitz matrix)}}
A matrix $A$ (finite or infinite) is Toeplitz if it is constant along its diagonals; that is, $A_{i,j} = A_{i+1,j+1}$
for any $i, j$ (where $i,j < N \in \mathbb{N}$ if $A$ is finite, or $i,j \in \mathbb{Z}_+$ when $A$ is infinite).
\end{definition}

Note that an infinite Toeplitz matrix induces an operator on $\ell^2(\mathbb{Z}_+)$.
We see from our example that if $f \in L^\infty$ is represented by the Fourier series
$$f(z) = \sum_{k=-\infty}^{\infty} c_k e^{i n \theta},$$
then it induces a Toeplitz matrix $T_f$ with $(T_f)_{j,k} = c_{j-k}$.
A natural question is to then ask whether a Toeplitz matrix induces a function, and the answer is affirmative. Firstly, we must define a relevant setting for our matrices; indeed, an important part of our
example was that $M_f \exp(2 \pi i j x)$ was well-defined. The following definitions are adapted from \parencite{arveson2002short}.

\begin{definition}{\textbf{(Hardy space)}}
Let $\zeta$ be the monomial function on $L^2(\mathbb{T})$, $\zeta(z) = z$, where $\mathbb{T}$ is the unit circle. Then the
Hardy space $H^2$ is the span of all non-negative exponents of $\zeta$; $\text{span}\{1, \zeta, \zeta^2...\}$.
\end{definition}

As we are on the unit circle, $z$ is more familiarly $e^{i \theta}$ for some $\theta$; then the basis of Hardy space becomes $\{e^{i n \theta}\}_{n \in \mathbb{Z}_+}$. Then we can identify any element $f \in H^2$ as any square-integrable function defined on the unit circle with the Fourier series 
$$f(e^{i \theta}) \sim \sum_{k=0}^\infty c_k e^{i k \theta},$$

that is, with all negative Fourier coefficients equal to zero. This can be identified with the operator on $\ell^2(\mathbb{Z}+)$ via the isomorphism $\sum_{k=0}^\infty c_k e^{i k \theta} \mapsto (c_k)_{k \in \mathbb{Z}_+}$ \parencite{bottcher2006analysis}.

 A generalised definition can be made for $H^p$ via any $L^p(\mathbb{T})$; we will almost entirely use $H^2$ (with the exception of needing $H^1$ later on), which is often defined as `the' Hardy space.

\begin{definition}{\textbf{(Toeplitz operator)}}
Let $\phi \in L^{\infty}(\mathbb{T})$ be bounded and measurable on the unit circle. The Toeplitz operator $T_\phi$ is the compression of $M_\phi$ 
to the Hardy space: $T_{\phi} = P_{H^2} M_\phi \big|_{H^2}$. We call $\phi$ the `symbol' of $T_\phi$.
\end{definition}

One may notice what appears like a clash of notation between the induced Toeplitz matrix that was just discussed with the Toeplitz operator. This is not
so; there is an elegant relation between Toeplitz operators and Toeplitz matrices.

\begin{theorem}
Let $A$ be a bounded operator on $H^2$ such that $(A \zeta^j, \zeta^k) = a_{j-k}$ for some sequence $(a_n)_{n \in \mathbb{Z}}$. Then there is
some function $\phi \in L^\infty$ such that $A = T_\phi$ and $a_n$ are the Fourier coefficients of $\phi$.
\end{theorem}
\begin{proof}
%%%%%TODO: DO WE PROVE THIS?
\end{proof}

Many properties of Toeplitz operators are hard to see via infinite matrices. Being able to represent them as both Fourier series and equally as the  
compressions of multiplication operators puts us on the firmer ground of functional analysis, rather than asymptotic linear algebra. From this, we
are now in the position to exactly calculate the spectrum of a Toeplitz operator with real-valued symbol.

\subsubsection{The spectrum of a Toeplitz operator with real symbol}

To begin, we require a pair of properties regarding functions in $L^1(\mathbb{T})$'s Hardy space, $H^1$. 

\begin{lemma}{\textbf{(Properties of $H^1$ functions)}}\label{thm:h1-properties}
Let $H^1$ be the space of all functions $f \in L^1(\mathbb{T})$ where $f$ has the Fourier series

$$f(e^{i \theta}) \sim \sum_{n=0}^\infty a_n e^{i n \theta}$$

i.e. has no negative Fourier coefficients. Then the following properties hold:
\begin{enumerate}
\item\label{item:h2-products-in-h1} If $f, g \in H^2$, then $fg \in H^1$;
\item\label{item:f-h1-const} If $f \in H^1$ and its complex conjugate $\overline{f}$ is also in $H^1$, then $f$ is constant.
\end{enumerate}
\end{lemma}
\begin{proof}
(\ref{item:h2-products-in-h1}.) Let $f, g$ be in $H^2$. Then in particular, $f, g \in L^2$, and so their product $fg$ is in $L^1$ (this can be seen directly by the H\"older inequality). We now look at the negative Fourier coefficients of
$fg$; these are $(fg, \zeta^n)$ for $n \in -\mathbb{N}$. If $f$ and $g$ are finite sums $\sum_{k=0}^N c_k \zeta^k$ then $fg$ is a finite sum of a similar 
form; the orthogonality of positive exponents of $\zeta$ on the unit circle then means that $(fg, \zeta^n) = 0$. Then by the continuity of scalar products,
the same holds for any $f, g \in H^2$, and so all the negative Fourier coefficients of $fg$ are zero; therefore $fg$ is an $L^1$ function with no negative 
Fourier coefficients and is in $H^1$.

(\ref{item:f-h1-const}.) For any function, we have the Fourier series
\begin{equation*}
f(e^{i \theta}) \sim \sum_{n=-\infty}^\infty c_n e^{i n \theta} \quad\text{and}\quad \overline{f(e^{i \theta})} \sim \sum_{n=-\infty}^\infty \overline{c_n} e^{-i n \theta}
\end{equation*}
and if $f \in H^1$, all of the negative Fourier coefficients must be zero; but from the Fourier series of $\overline{f}$ we can see these are exactly
the positive Fourier coefficients of $\overline{f}$; vice versa, all of the negative coefficients of $\overline{f}$ are zero and these are the positive
coefficients of $f$! This only leaves the coefficient for the constant term; thus $f$ has the Fourier series of a constant function, so must itself
be a constant function in $L^1$.
\end{proof}

\begin{theorem}{\textbf{(Hartman-Wintner)}}
Let $\phi \in L^\infty$ be real-valued. Then $\Spec(T_\phi) = [m, M]$, where $m$ and $M$ are the infimum and supremum of the essential range of $\phi$ (defined in Definition \ref{defn:essential-range}) respectively.
\end{theorem}
\begin{proof}
Our proof will consist of a series of claims about the resolvent set of $T_\phi$, following a proof outlined in an exercise of Arveson \parencite{arveson2002short} (Chapter 4.6, exercises 2-5). 
Note we already know that as $\phi$ is real-valued, $T_\phi$ is self-adjoint, and so its spectrum is on the real line. 
Furthermore, we will assume that $\phi$ is non-constant, as if $\phi$ is constant then $T_\phi$ is some constant multiple of the identity operator
and its spectrum is simply the set containing that constant, and the result holds.
\begin{enumerate}[I.]
\item Let $\lambda \in \mathbb{R}$ be such that $T_\phi - \lambda$ is invertible. Then there is a non-zero function $f \in H^2$ such that $(T_\phi - \lambda)f(z) = \underline{1}$, where
$\underline{1}$ is the constant function in $H^2$; $\underline{1} : z \mapsto k$ for some $k \in \mathbb{C}.$ By the definition of invertibility this is true for $f$ = $(T_\phi - \lambda)^{-1}\underline{1}$.

\item Now we claim that $(\phi - \lambda)|f|^2$ is constant almost everywhere. To do this, we first show that it is in $H^1$. Indeed, we have
$(\phi - \lambda)|f|^2 = ((\phi - \lambda)\overline{f}) f$. Then by the previous part, $$(\phi - \lambda)\overline{f} = \overline{(\phi - \lambda)f} = \overline{\underline{1}}$$
where $\overline{\underline{1}}$ maps $z$ to $\overline{k}$. Then it is still a constant function so is still in $H^2$, and so by Lemma \ref{thm:h1-properties}.\ref{item:h2-products-in-h1}, $((\phi - \lambda)\overline{f}) f$ is in $H^1$. Now we see that because $\phi$ and $\lambda$ are real-valued, $(\phi - \lambda)|f|^2 = \overline{(\phi - \lambda)|f|^2}$, so $\overline{(\phi - \lambda)|f|^2} \in H^1$. Then by Lemma \ref{thm:h1-properties}.\ref{item:f-h1-const}, $(\phi - \lambda)|f|^2$ is constant; let this constant be called $c$.
\item Our next claim is that $\phi - \lambda$ crosses the $x$-axis almost nowhere. This is immediate once we invoke a theorem of F. and M. Riesz\footnote{Which can be proven as a corollary of a famous theorem of Buerling; see \parencite{arveson2002short}, chapter 4.5.}, which states
\begin{displayquote}
\emph{Let $f$ be a non-zero function in $H^2$. Then the set $\{z \in \mathbb{T} : f(z) = 0\}$ has Lebesgue measure zero.}
\end{displayquote}
This means that $(\phi(z) - \lambda) = \frac{c}{|f(z)|^2}$ is well-defined on $L^\infty$. Then because $|f|^2 > 0$ a.e., $(\phi - \lambda)$ is positive almost everywhere if $c > 0$, and negative almost everywhere if $c < 0$. Note that if $c = 0$, then $(\phi(z) - \lambda) = 0$, so $\phi$ is the 
constant function with value $\lambda$ and the result holds by our remark at the beginning of this proof.
\item Finally, we show $\Spec(T_\phi) = [m, M]$. As we have seen, if $T_\phi - \lambda$ is invertible, then either:
\begin{itemize}
\item $\phi(z) - \lambda < 0$ a.e., so $\phi(z) < \lambda$ a.e., so $\lambda > M$, or
\item $\phi(z) - \lambda > 0$ a.e., so by the same argument $\lambda < m$.
\end{itemize}
Thus $T_\phi - \lambda$ is invertible only outside of the set $[m, M]$, as required.
\end{enumerate}
\end{proof}

\subsubsection{Multiplication operators, Toeplitz operators, and spectral pollution}
Let us now bring the discussion back to spectral pollution, and to our discovery in Example \ref{exp:mult-op-toeplitz}. The Ritz matrices corresponding to the
multiplication operator $M_f$ is \emph{identical} to the Ritz matrices corresponding to the Toeplitz operator $T_f$ (which are just truncations of the 
corresponding Toeplitz matrix), but these operators have different spectra. No wonder we are seeing extra eigenvalues in the gap; the approximation
`cannot tell' the difference between an operator which has the essential range as its spectrum from one which has the same maximum and minimum but with
all gaps filled in. Heuristically, one may even expect that with a large enough approximation, the \emph{entire gap} could fill up with spurious
eigenvalues.\\

More rigorously, it is possible to show that for any point in a gap of the multiplication operator's spectrum, some subsequence of truncations will have 
an approximate eigenvalue converging to that point. 

\begin{theorem}{\textbf{(Schmidt-Spitzer \parencite{schmidt1960toeplitz})}}
Let $T_n$ be the $n \times n$ matrix from taking the first $n$ rows and columns of a Toeplitz operator $T_f$. Consider the set
$$B = \{\lambda \in \mathbb{C} : \lambda = \lim_{m \rightarrow \infty} \lambda_{i_m}, \lambda_{i_m} \in \Spec(T_{i_m}), i_m \rightarrow \infty \}.$$
where $i_m$ is a subsequence of $\mathbb{N}$. Then if $f$ is real-valued, $B = \Spec(T_f)$.

\end{theorem}

\printbibliography
\end{document}
