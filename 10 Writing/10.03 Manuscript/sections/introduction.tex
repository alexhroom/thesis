\documentclass[../main.tex]{subfiles}

\begin{document}
\section{Introduction}
The computation of spectra can boldly be considered the 'fundamental problem of operator theory' \parencite{arveson2002short}. The spectrum
of an operator holds the same key to the entire structure as eigenvalues do for linear algebra, however (as often occurs when passing from 
the finite- to infinite-dimensional case) we lose the ease of representation that allows the creation of such algorithms and formulae as those 
used for matrices. 

We must first define our quantity of interest: the spectrum of an operator.
\begin{definition}{\textbf{(Resolvent and spectrum)}}\index{resolvent}\index{spectrum}
(Adapted from \parencite{evans2010partial}) Let $T$ be a linear operator on a Banach space.
\begin{itemize}
\item The resolvent of $T$ is the set $\rho(T) := \{\eta \in \mathbb{C} : (T - \eta I)\text{ is bijective}\}$, where I is the identity operator. 
\item The spectrum of $T$, denoted $\Spec(T)$, is $\mathbb{C} \setminus \rho(T)$, i.e. the set of all complex numbers $\lambda$ such 
that the operator $(T - \lambda I)$ does not have a bounded inverse.
\end{itemize}
\end{definition}

One can see that this concept generalises the eigenvalues of a matrix to any Banach space, and that for a finite-dimensional Banach space
(of which $\mathbb{R}^n$ and $\mathbb{C}^n$ are major examples) we can recover the eigenvector equation. For an operator on an
infinite-dimensional space, values which satisfy the eigenvector equation \emph{are} in the spectrum of the operator, but they do not
comprise the entire spectrum - nor do we necessarily have a spectrum made up of a discrete set of values.

\begin{example}
Let $S$ be the 'right-shift' operator on the sequence space $\ell^2(\mathbb{N})$, which has the following action:
$S(x_1, x_2, x_3, ...)$ = $(0, x_1, x_2, x_3, ...)$; that is, for a sequence $u$ we map $u_n$ to $u_{n+1}$, and pad the vector with a zero in $u_1$'s place.
$S$ has no eigenvalues, and its spectrum is the open unit disc $D = \{z \in \mathbb{C} : |z| = 1\}$.
\end{example}
\begin{proof}{\emph{(sketch)}}
For the lack of eigenvalues; we see that if $Su = \lambda u$, 
$$(0, u_1, u_2, u_3 ...) = (\lambda u_1, \lambda u_2, \lambda u_3 ...)$$
so in particular, we would require $\lambda u_1 = 0$. Then as $\lambda \neq 0$, $u_1 =0$. Then $\lambda u_2 = u_1 = 0$, and continuing this we
can see that $u_n = 0$ for all $n$, i.e. $u$ must be the zero vector. Thus there is no non-zero vector satisfying the eigenvector equation.

To calculate its spectrum, we make use of a natural relation between the spectrum of an operator T and the spectrum of its adjoint: 
\begin{align*}
\lambda \in \Spec(T)  &\text{ iff } (T - \lambda)\text{ is not invertible }\\
& \text{ iff } (T - \lambda)^*\text{ is not invertible (this is true for any operator)}\\
& = (T^* - \overline{\lambda}) \text{ not invertible, i.e. } \overline{\lambda} \in \Spec(T^*).
\end{align*}
in this case, the adjoint is the left-shift operator $S^*u = (u_2, u_3, u_4, ...).$; we can see that for any value $\lambda$ we have for the vector
$(\lambda, \lambda^2, \lambda^3...)$
$$S^*(\lambda, \lambda^2, \lambda^3) = (\lambda^2, \lambda^3, \lambda^4, ...) = \lambda(\lambda, \lambda^2, \lambda^3...)$$
which is in $\ell^2(\mathbb{N})$ and is thus an eigenvector iff $|\lambda| < 1$; hence the unit disc is in $\Spec(S^*)$ and hence its
complex conjugate (which is also the unit disc) is in $\Spec(S)$.
\end{proof}

As we have alluded to, there is no universal algorithm for the calculation of operator spectra in the way that there is the QR algorithm (\parencite{francis1961qr}) for matrices. To devise a formula for the spectrum of even a specific subset of a class of operators is a mathematical
feat, and varieties of operators important to fields such as quantum physics (\cite{lewin2010spectral}), hydrodynamics (\cite{manning2008descriptor}), and crystallography (\cite{cances2012periodic}) still withhold the structure of their spectra from decades-long attempts at discovery. 
To this end, we must employ numerical methods.
We shall find that even approximating spectra computationally is not so easy.

\begin{definition}{\textbf{(Compressions, truncations, and Ritz matrices)}}\index{compression}\index{truncation}\index{matrix!Ritz}
(Adapted from \parencite{davies1995spectral})
Let $T$ be an operator on a Hilbert space $H$, $\mathcal{L} \subseteq Dom(T)$ a closed linear subspace, and $P_\mathcal{L}$ the orthogonal projection
of $H$ onto $\mathcal{L}$.
\begin{itemize}
\item The \textbf{compression} of the operator $T$, which we will often denote $T_\mathcal{L}$, is defined
$$T_\mathcal{L} \eqdef P_\mathcal{L} T\big|_{\mathcal{L}}$$
where $\big|_{\mathcal{L}}$ denotes domain restriction to $\mathcal{L}$.
\item If $\{\phi_n\}_{n \in \mathbb{N}}$ is an orthonormal basis for $H$, the n'th \textbf{truncation} of $T$ is the compression of $T$ to $\text{Span}\{\phi_1, \phi_2, ..., \phi_n\}$.
\item We will call the matrix representation of $T$ truncated to $\text{Span}\{\phi_1, \phi_2, ..., \phi_n\}$ the \textbf{Ritz matrix} of $T$,
and denote it $T_n$ when the context is obvious. $T_n$ is an $n \times n$ matrix with entries
$$(T_n)_{i,j}  \eqdef (T\phi_i, \phi_j) \quad \forall i, j \leq n.$$
\end{itemize}
\end{definition}

This definition raises a natural question; do the eigenvalues of the matrices $T_n$ converge to the spectrum of the operator $T$? This question was investigated by Walther H. W. Ritz for whom we name our matrices, as well as by Boris Galerkin; the method of approximating the spectrum of an operator by the eigenvalues of its truncations is often called the Ritz method, Galerkin method, or indeed the Ritz-Galerkin method. We will formulate answers
to this question in due course, but for our introduction it suffices to say that the answer is "not quite". What can go wrong?

\begin{definition}{\textbf{(Spectral pollution)}}\index{spectrum!pollution of}
(Adapted from \parencite{davies1995spectral})
Let $(T_n)_{n \in \mathbb{N}}$ be an increasing sequence of truncations of an operator $T$. A value $\lambda \in \mathbb{C}$ is said to be a point of \textbf{spectral pollution} if there is a sequence $\lambda_n \in \Spec(T_n)$ such that $\lambda_n \rightarrow \lambda$ but $\lambda \notin \Spec(A)$.
\end{definition}

Points of spectral pollution are, intuitively, artefacts of the approximation which will never converge to a point in the actual spectrum. We will see that they exist, that they are relatively common, and that they \emph{get worse} as the approximation goes to higher iterations. Unless we already
know what the spectrum of the operator is, it can be incredibly hard for us to decide whether a point is actually in the spectrum or whether it is
spurious. In applications of spectral theory, this difference can be beyond a simple 'error bar' nuisance - rather, a confounding problem.

\begin{example}
%%% TODO: SCHRODINGER EXAMPLE?
\end{example}

\end{document}