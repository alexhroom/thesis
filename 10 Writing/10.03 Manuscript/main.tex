\documentclass{article}
\setlength{\parskip}{0.75ex}

%% page formatting
\usepackage[T1]{fontenc}
\usepackage[utf8]{inputenc}
\usepackage{graphicx}
\usepackage[margin=1in]{geometry}
\usepackage[justification=centering]{caption}
\usepackage{subcaption}
\usepackage{blindtext}
\usepackage{enumerate}
\usepackage{csquotes}

%% bibliography and index
\usepackage[style=numeric, sorting=none]{biblatex}
\usepackage{imakeidx}
\addbibresource{bibliography.bib}
\makeindex

%% prettification definitions
\usepackage[x11names]{xcolor}
\definecolor{IKB}{HTML}{002FA7}

%% section header formatting
\usepackage[explicit]{titlesec}
\titleformat{\section}[hang]{\newpage\fontfamily{put}\color{IKB}\Large\scshape\bfseries}{\Roman{section}}{0.61803cm}{\rule[-0.25cm]{1.5pt}{1cm}\quad#1}

%% hyperlink formatting
\usepackage{hyperref}
\hypersetup{
    colorlinks=true,
    linkcolor=black,
    urlcolor=blue,
    citecolor=black
}

%% theorem definitions
\usepackage{amsthm}
\newtheorem*{definition}{Definition}
\newtheorem*{remark}{Remark}
\newtheorem{example}{Example}
\newtheorem{theorem}{Theorem}[section]
\newtheorem{lemma}[theorem]{Lemma}
\newtheorem*{lemma*}{Lemma}
\newtheorem{proposition}[theorem]{Proposition}
\newtheorem{corollary}[theorem]{Corollary}

%% symbol definitions
\usepackage{mathtools}
\usepackage{amsmath}
\usepackage{amssymb}
\newcommand{\Spec}{\mathrm{Spec}} % spectrum
\newcommand{\Dom}{\mathrm{Dom}} % domain
\newcommand{\Ran}{\mathrm{Ran}} % range
\newcommand{\Ker}{\mathrm{Ker}} % kernel
\newcommand{\rank}{\mathrm{rank}} % rank
\newcommand{\Num}{\mathrm{W}} % numerical range
\newcommand{\supp}{\mathrm{supp}} % support
\newcommand{\conv}{\mathrm{conv}} % convex hull
\newcommand{\sobolev}{W^{1, 2}} % sobolev space H^1 (we avoid conflict with Hardy space notation)
\newcommand{\hilbert}{\mathcal{H}} % Hilbert space generic symbol
\newcommand{\diag}{\mathrm{diag}} % Diagonal infinite matrix

\newcommand{\1}{\mathbf{1}} % characteristic function
\newcommand{\eqdef}{\mathrel{\mathop:}=} % 'defined by' symbol

\usepackage{subfiles}
\graphicspath{{images/}{../images/}}

\begin{document}
  \begin{titlepage}
    \vspace*{\fill}
    \begin{center}
      {\Huge Spectral Pollution}\\[0.5cm]
      {\Large Alex H. Room}\\[0.4cm]
      {\Large \today}
    \end{center}
     \vspace{\fill}
     {\normalsize\textbf{Abstract} {\blindtext}}
     \vspace*{\fill}
  \end{titlepage}
\tableofcontents
\subfile{sections/introduction}

\subfile{sections/multiplication-operator}

\subfile{sections/jacobi-matrices}

\subfile{sections/pollution-bounds}

\subfile{sections/detecting-pollution}

\appendix
\titleformat{\section}[hang]{\newpage\fontfamily{put}\color{IKB}\Large\scshape\bfseries}{\thesection}{0.61803cm}{\rule[-0.25cm]{1.5pt}{1cm}\quad#1}
\section{Note on numerical examples}\label{sec:numerical-note}
All numerical examples in this document were calculated using the Python library \texttt{specpol}, written
by the author. 

\clearpage
\printbibliography
\printindex
\end{document}
