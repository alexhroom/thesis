\documentclass{article}

%% page formatting
\usepackage[utf8]{inputenc}
\usepackage{graphicx}
\usepackage[margin=1in]{geometry}
\usepackage[justification=centering]{caption}
\usepackage{blindtext}
\usepackage{fix-cm}

%% bibliography
\usepackage[style=numeric, sorting=none]{biblatex}
\addbibresource{bibliography.bib}

%% extra colours
\usepackage[x11names]{xcolor}
\definecolor{IKB}{HTML}{002FA7}

%% section header formatting
\usepackage[explicit]{titlesec}
\renewcommand{\thesection}{\Roman{section}}
\titleformat{\section}[hang]{\newpage\color{IKB}\Large\scshape\bfseries}{\thesection}{0.61803cm}{\rule[-0.25cm]{1.5pt}{1cm}\quad#1}

%% hyperlink formatting
\usepackage{hyperref}
\hypersetup{
    colorlinks=true,
    linkcolor=black,
    urlcolor=blue,
    citecolor=black
}

%% theorem definitions
\usepackage{amsthm}
\newtheorem*{definition}{Definition}
\newtheorem{example}{Example}

%% symbol definitions
\usepackage{amsmath}
\usepackage{amssymb}
\newcommand{\Spec}{\text{Spec}} % spectrum
\newcommand{\Dom}{\text{Dom}} % domain
\newcommand{\Ran}{\text{Ran}} % range
\newcommand{\1}{\mathbf{1}} % characteristic fucntion

\begin{document}
  \begin{titlepage}
    \vspace*{\fill}
    \begin{center}
      {\Huge Title}\\[0.5cm]
      {\Large Alex H. Room}\\[0.4cm]
      {\Large \today}
    \end{center}
     \vspace{\fill}
     {\normalsize\textbf{Abstract} {\blindtext}}
     \vspace*{\fill}
  \end{titlepage}
\tableofcontents


\section{Introduction}
The computation of spectra can be boldly considered the 'fundamental problem of operator theory' \parencite{arveson2002short}; 

\section{Spectra}
We must first discuss our quantity of interest: the spectrum of an operator.
\begin{definition}{\textbf{(Resolvent and spectrum)}}
(Adapted from \parencite{evans2010partial}) Let $T$ be a linear operator on a Banach space.
The resolvent of $T$ is the set $\rho(T) := \{\eta \in \mathbb{C} : (T - \eta I)\text{ is bijective}\}$, where I is the identity operator. 

The spectrum of $T$, denoted
$\Spec(T)$, is $\mathbb{C} \setminus \rho(T)$, i.e. the set of all complex numbers $\lambda$ such that the operator $(T - \lambda I)$ does not have a bounded inverse.
\end{definition}
Indeed, if the Banach space is finite-dimensional, this definition coincides with that of the eigenvalues of a matrix. An infinite-dimensional operator also has eigenvalues, but these are just a subset of the whole spectrum (and sometimes can even be none of the spectrum), as we will see below. One can very simply check that if a non-zero vector $x$ satisfies $Tx = \lambda x$, then $\lambda$ must be in $\Spec(T)$.

\begin{example}
Let $M_f$ denote the multiplication operator by a function $f$ on $L^2(0, 1)$; this operator has action $M_f u(x)$ = $f(x)u(x)$. If $f$ is continuous, then $\Spec(M_f)$ is equal to the range of $f$.
\end{example}
\begin{proof}
Let the range of $f$ be denoted $\Ran(f)$, and consider the function $g = \frac{1}{f(x) - \lambda}$. If $\lambda \notin \Ran(f)$, $g$ is bounded, and so one can see that $M_g$ is an inverse to $M_f - \lambda I$:
$$M_g(M_f - \lambda I)u(x) = M_g(f(x) - \lambda)u(x) = \frac{f(x) - \lambda}{f(x) - \lambda}u(x) = u(x).$$

Now take $\lambda \in \Ran(f)$. Then 
\end{proof}

\printbibliography
\end{document}