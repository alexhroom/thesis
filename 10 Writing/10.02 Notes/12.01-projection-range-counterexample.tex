\documentclass[a5paper]{article}

\usepackage[utf8]{inputenc}
\usepackage[T1]{fontenc}
\usepackage{graphicx}
\usepackage[margin=0.6in]{geometry}

\usepackage{amsmath}
\usepackage{amssymb}

\usepackage{amsthm}
\newtheorem*{proposition}{Proposition}

\usepackage{hyperref}
\hypersetup{
    colorlinks=true,
    linkcolor=black,
    urlcolor=blue,
    citecolor=black
}

\newcommand{\Num}{\text{Num}}

\begin{document}
\begin{proposition}
Let $T$ be an operator on a Hilbert space $H$, and $P: H \rightarrow \mathcal{L}$ an orthogonal projection
to a proper closed subspace $\mathcal{L} \subset H$. For the operator $PTP: H \rightarrow \mathcal{L}$ (that
is, $PTP$ \textbf{not} restricted to $\mathcal{L}$), the inclusion $\Num(PTP) \subseteq \Num(T)$ does not hold.
\end{proposition}
\begin{proof}

As a counterexample, let $I$ be the identity operator $u \mapsto u$ on $H$. Then
\begin{equation*}
\begin{split}
\Num(I)  & = \{\langle Iu, u \rangle : u \in H, \|u\| = 1\} \\
& =  \{\langle u, u \rangle : u \in H, \|u\| = 1\} \\
& =  \{\|u\|^2 : u \in H, \|u\| = 1\} \\
& = \{1\}.
\end{split}
\end{equation*}

Now let $P$ be the orthogonal projection in the proposition.
Without loss of generality, we can take $v \in \mathcal{L}^\perp$ such that $\|v\| = 1$; as $\mathcal{L}^\perp$ is also a linear subspace and $\mathcal{L}$ is proper, there is some non-zero element $v$ in $\mathcal{L}^\perp$ and
we may normalise it to $\frac{v}{\|v\|} \in \mathcal{L}^\perp$. Then we note $Pv = 0$ and thus
$$\langle PIPv, v \rangle = \langle P(0), v \rangle =  \langle 0, v \rangle = 0,$$ 
so $0 \in \Num(PIP)$. But $0 \notin \Num(I) = \{1\}$ and so $\Num(PIP) \nsubseteq \Num(I)$.

\end{proof}

\end{document}